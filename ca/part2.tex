\input{preamble.tex}
\date{10 de maig de 2024}

\subtitle{Segona part}

\begin{document}

%%%%%%%%%%%%%%%%%%%%%%%%%%%%%%%%%%%%%%%%%%%%%%%%%%%%%%%%%%%%%%%%%%%%%%%%%%%%%%%
%%%%%%%%%%%%%%%%%%%%%%%%%%%%%%%%%%%%%%%%%%%%%%%%%%%%%%%%%%%%%%%%%%%%%%%%%%%%%%%
%%%%%%%%%%%%%%%%%%%%%%%%%%%%%%%%%%%%%%%%%%%%%%%%%%%%%%%%%%%%%%%%%%%%%%%%%%%%%%%
\begin{frame}
\titlepage
\end{frame}

%%%%%%%%%%%%%%%%%%%%%%%%%%%%%%%%%%%%%%%%%%%%%%%%%%%%%%%%%%%%%%%%%%%%%%%%%%%%%%%
%%%%%%%%%%%%%%%%%%%%%%%%%%%%%%%%%%%%%%%%%%%%%%%%%%%%%%%%%%%%%%%%%%%%%%%%%%%%%%%
%%%%%%%%%%%%%%%%%%%%%%%%%%%%%%%%%%%%%%%%%%%%%%%%%%%%%%%%%%%%%%%%%%%%%%%%%%%%%%%
\section{Dubtes classe anterior}
\begin{frame}{Continguts}
\begin{multicols}{2}
\tableofcontents[currentsection]
\end{multicols}
\end{frame}

%%%%%%%%%%%%%%%%%%%%%%%%%%%%%%%%%%%%%%%%%%%%%%%%%%%%%%%%%%%%%%%%%%%%%%%%%%%%%%%
%%%%%%%%%%%%%%%%%%%%%%%%%%%%%%%%%%%%%%%%%%%%%%%%%%%%%%%%%%%%%%%%%%%%%%%%%%%%%%%
%%%%%%%%%%%%%%%%%%%%%%%%%%%%%%%%%%%%%%%%%%%%%%%%%%%%%%%%%%%%%%%%%%%%%%%%%%%%%%%
\begin{frame}[fragile]{Dubtes classe anterior}
\begin{itemize}
    \item \textbf{Com s'escapa una seqüència sencera?}\\
    Cal escapar la pròpia barra inversa \keystroke{\textbackslash} amb \cmd{\textbackslash{}textbackslash{}}.

\begin{exampletwouptiny}
% podem separar l'ordre amb espai
\textbackslash includegraphics

% o amb claus
\textbackslash{}includegraphics
\end{exampletwouptiny}
\item \textbf{Com s'evita que es tallin paraules amb guió a final de línia?}
Es pot posar el següent al preàmbul (\href{https://tex.stackexchange.com/questions/5036/how-to-prevent-latex-from-hyphenating-the-entire-document}{font}):

\begin{exampletiny}
\tolerance=1
\emergencystretch=\maxdimen
\hyphenpenalty=10000
\hbadness=10000
\end{exampletiny}
\end{itemize}
\end{frame}


%%%%%%%%%%%%%%%%%%%%%%%%%%%%%%%%%%%%%%%%%%%%%%%%%%%%%%%%%%%%%%%%%%%%%%%%%%%%%%%
%%%%%%%%%%%%%%%%%%%%%%%%%%%%%%%%%%%%%%%%%%%%%%%%%%%%%%%%%%%%%%%%%%%%%%%%%%%%%%%
%%%%%%%%%%%%%%%%%%%%%%%%%%%%%%%%%%%%%%%%%%%%%%%%%%%%%%%%%%%%%%%%%%%%%%%%%%%%%%%
\section{Documents Estructurats}
\begin{frame}{Continguts}
\begin{multicols}{2}
\tableofcontents[currentsection]
\end{multicols}
\end{frame}

%%%%%%%%%%%%%%%%%%%%%%%%%%%%%%%%%%%%%%%%%%%%%%%%%%%%%%%%%%%%%%%%%%%%%%%%%%%%%%%
%%%%%%%%%%%%%%%%%%%%%%%%%%%%%%%%%%%%%%%%%%%%%%%%%%%%%%%%%%%%%%%%%%%%%%%%%%%%%%%
%%%%%%%%%%%%%%%%%%%%%%%%%%%%%%%%%%%%%%%%%%%%%%%%%%%%%%%%%%%%%%%%%%%%%%%%%%%%%%%
\subsection{Títol i Resum}
\begin{frame}[fragile]{Títol i Resum}
\begin{itemize}{\small
\item A \LaTeX{}, \cmdbs{title}, \cmdbs{author} i \cmdbs{date} es defineixen al preàmbul.
\item Dins el document, \cmdbs{maketitle} crea el títol.
\item Utilitza l'entorn \bftt{abstract} per crear un resum.
\item El paquet \bftt{babel} defineix l'idioma del document.
}\end{itemize}
\begin{minipage}{0.55\linewidth}
\inputminted[fontsize=\scriptsize,frame=single,resetmargins]{latex}%
  {structure-title.tex}
\end{minipage}
\begin{minipage}{0.35\linewidth}
\includegraphics[width=1.3\textwidth,clip,trim=2.4in 7in 2.4in 2in]{structure-title.pdf}
\end{minipage}
\end{frame}

%%%%%%%%%%%%%%%%%%%%%%%%%%%%%%%%%%%%%%%%%%%%%%%%%%%%%%%%%%%%%%%%%%%%%%%%%%%%%%%
%%%%%%%%%%%%%%%%%%%%%%%%%%%%%%%%%%%%%%%%%%%%%%%%%%%%%%%%%%%%%%%%%%%%%%%%%%%%%%%
%%%%%%%%%%%%%%%%%%%%%%%%%%%%%%%%%%%%%%%%%%%%%%%%%%%%%%%%%%%%%%%%%%%%%%%%%%%%%%%
\subsection{Seccions}
\begin{frame}{Seccions}
\begin{itemize}{\small
    \item \cmdbs{section} i \cmdbs{subsection}
        (\href{https://www.overleaf.com/learn/latex/Sections_and_chapters}{hi ha més nivells})
\item Què creus que fan \cmdbs{section*} i \cmdbs{subsection*}?
}\end{itemize}
\begin{minipage}{0.55\linewidth}
\inputminted[fontsize=\scriptsize,frame=single,resetmargins]{latex}%
  {structure-sections.tex}
\end{minipage}
\begin{minipage}{0.35\linewidth}
\includegraphics[width=\textwidth,clip,trim=1.5in 6in 4in 1in]{structure-sections.pdf}
\end{minipage}
\end{frame}

%%%%%%%%%%%%%%%%%%%%%%%%%%%%%%%%%%%%%%%%%%%%%%%%%%%%%%%%%%%%%%%%%%%%%%%%%%%%%%%
%%%%%%%%%%%%%%%%%%%%%%%%%%%%%%%%%%%%%%%%%%%%%%%%%%%%%%%%%%%%%%%%%%%%%%%%%%%%%%%
%%%%%%%%%%%%%%%%%%%%%%%%%%%%%%%%%%%%%%%%%%%%%%%%%%%%%%%%%%%%%%%%%%%%%%%%%%%%%%%
\subsection{Etiquetes i Referències Creuades}
\begin{frame}[fragile]{Etiquetes i Referències Creuades}
\begin{itemize}{\small
\item \cmdbs{label} serveix per etiquetar i \cmdbs{ref} per referenciar.
\item Les dues serveixen per seccions, figures, taules \dots
}\end{itemize}
\begin{minipage}{0.55\linewidth}
\inputminted[fontsize=\scriptsize,frame=single,resetmargins]{latex}%
  {structure-crossref.tex}
\end{minipage}
\begin{minipage}{0.35\linewidth}
\includegraphics[width=\textwidth,clip,trim=1.5in 6in 4in 1in]{structure-crossref.pdf}
\end{minipage}
\end{frame}

%%%%%%%%%%%%%%%%%%%%%%%%%%%%%%%%%%%%%%%%%%%%%%%%%%%%%%%%%%%%%%%%%%%%%%%%%%%%%%%
%%%%%%%%%%%%%%%%%%%%%%%%%%%%%%%%%%%%%%%%%%%%%%%%%%%%%%%%%%%%%%%%%%%%%%%%%%%%%%%
%%%%%%%%%%%%%%%%%%%%%%%%%%%%%%%%%%%%%%%%%%%%%%%%%%%%%%%%%%%%%%%%%%%%%%%%%%%%%%%
\begin{frame}[fragile]{\protect\bftt{cleveref}: Millors Referències Creuades (opinió)}
\begin{itemize}{\small
\item \bftt{cleveref} permet tenir automàticament el nom del que estiguem referenciant.
\item Només cal canviar \cmdbs{ref} per \cmdbs{cref}, i serveix per seccions, figures i taules.
}\end{itemize}
\begin{minipage}{0.55\linewidth}
\inputminted[fontsize=\scriptsize,frame=single,resetmargins]{latex}%
  {structure-cleveref.tex}
\end{minipage}
\begin{minipage}{0.35\linewidth}
\includegraphics[width=\textwidth,clip,trim=1.5in 6in 4in 1in]{structure-cleveref.pdf}
\end{minipage}
\end{frame}

%%%%%%%%%%%%%%%%%%%%%%%%%%%%%%%%%%%%%%%%%%%%%%%%%%%%%%%%%%%%%%%%%%%%%%%%%%%%%%%
%%%%%%%%%%%%%%%%%%%%%%%%%%%%%%%%%%%%%%%%%%%%%%%%%%%%%%%%%%%%%%%%%%%%%%%%%%%%%%%
%%%%%%%%%%%%%%%%%%%%%%%%%%%%%%%%%%%%%%%%%%%%%%%%%%%%%%%%%%%%%%%%%%%%%%%%%%%%%%%
\subsection{Exercici}
\begin{frame}[fragile]{Exercici Documents Estructurats}

\begin{block}{Escriu el següent article amb \LaTeX
\footnote{De \url{http://pdos.csail.mit.edu/scigen/},
un generador d'articles aleatoris.}:}
\begin{center}
\fbox{\href{\fileuri/structure-exercise-solution.pdf}{Clica per obrir l'article}}
\end{center}
Fes que l'article tingui la mateixa pinta.
Utilitza \cmdbs{ref} per evitar escriure explícitament la secció i el número d'exemple.
\end{block}
\vskip 2ex
\begin{center}
\fbox{\href{\wlnewdoc{structure-exercise.tex}}{%
Clica per obrir l'exercici a \wllogo{}}}
\end{center}

\begin{itemize}
\item Un cop ho hagis provat,
\fbox{\href{\wlnewdoc{structure-exercise-solution.tex}}{%
Clica per veure la solució}}.
\end{itemize}
\end{frame}


%%%%%%%%%%%%%%%%%%%%%%%%%%%%%%%%%%%%%%%%%%%%%%%%%%%%%%%%%%%%%%%%%%%%%%%%%%%%%%%
%%%%%%%%%%%%%%%%%%%%%%%%%%%%%%%%%%%%%%%%%%%%%%%%%%%%%%%%%%%%%%%%%%%%%%%%%%%%%%%
%%%%%%%%%%%%%%%%%%%%%%%%%%%%%%%%%%%%%%%%%%%%%%%%%%%%%%%%%%%%%%%%%%%%%%%%%%%%%%%
\section{Bibliografies}
\begin{frame}{Continguts}
\begin{multicols}{2}
\tableofcontents[currentsection]
\end{multicols}
\end{frame}

%%%%%%%%%%%%%%%%%%%%%%%%%%%%%%%%%%%%%%%%%%%%%%%%%%%%%%%%%%%%%%%%%%%%%%%%%%%%%%%
%%%%%%%%%%%%%%%%%%%%%%%%%%%%%%%%%%%%%%%%%%%%%%%%%%%%%%%%%%%%%%%%%%%%%%%%%%%%%%%
%%%%%%%%%%%%%%%%%%%%%%%%%%%%%%%%%%%%%%%%%%%%%%%%%%%%%%%%%%%%%%%%%%%%%%%%%%%%%%%
\subsection{Bib\LaTeX{}}
\begin{frame}[fragile]{Bib\LaTeX{} 1}
\begin{itemize}
\item Poseu les referències a un fitxer \bftt{.bib} en format `bibtex':
\vskip 1ex
\inputminted[fontsize=\scriptsize,frame=single]{latex}{bib-example.bib}
\item La majoria de gestors i buscadors d'articles ofereixen aquesta opció.
\end{itemize}
\end{frame}

%%%%%%%%%%%%%%%%%%%%%%%%%%%%%%%%%%%%%%%%%%%%%%%%%%%%%%%%%%%%%%%%%%%%%%%%%%%%%%%
%%%%%%%%%%%%%%%%%%%%%%%%%%%%%%%%%%%%%%%%%%%%%%%%%%%%%%%%%%%%%%%%%%%%%%%%%%%%%%%
%%%%%%%%%%%%%%%%%%%%%%%%%%%%%%%%%%%%%%%%%%%%%%%%%%%%%%%%%%%%%%%%%%%%%%%%%%%%%%%
\begin{frame}[fragile]{Bib\LaTeX{} 2}
\begin{itemize}
\item Cada entrada del fitxer \bftt{.bib} té una clau que serveix per referenciar-la dins el text.
Per exemple, \bftt{Jacobson1999Towards} és la clau d'aquest article:
\begin{minted}[fontsize=\small,frame=single]{latex}
@Article{Jacobson1999Towards,
  author = {Van Jacobson},
  ...
}
\end{minted}
\item És bona idea canviar la clau que trobeu per defecte a una que contingui nom any i títol.
\item \LaTeX{} formatejarà automàticament les cites en el format adequat (Science, APA, Chicago \dots), tant dins al text com a l'apartat de referències.
\end{itemize}
\end{frame}

%%%%%%%%%%%%%%%%%%%%%%%%%%%%%%%%%%%%%%%%%%%%%%%%%%%%%%%%%%%%%%%%%%%%%%%%%%%%%%%
%%%%%%%%%%%%%%%%%%%%%%%%%%%%%%%%%%%%%%%%%%%%%%%%%%%%%%%%%%%%%%%%%%%%%%%%%%%%%%%
%%%%%%%%%%%%%%%%%%%%%%%%%%%%%%%%%%%%%%%%%%%%%%%%%%%%%%%%%%%%%%%%%%%%%%%%%%%%%%%
\begin{frame}[fragile]{Bib\LaTeX{} 3}
\begin{itemize}
\item Feu servir el paquet \bftt{biblatex}\footnote{Hi ha un altre paquet (més vell) de referència, \bftt{natbib}. És possible que us el trobeu a alguna plantilla. Fa servir \cmdbs{citet} i \cmdbs{citep}.}
    amb \cmdbs{cite} (genèric), \cmdbs{citetext} (textual) i \cmdbs{parencite} (entre parèntesis).
\item Al final del document poseu \cmdbs{printbibliography}
\begin{minipage}{0.55\linewidth}
\inputminted[fontsize=\scriptsize,frame=single,resetmargins]{latex}%
  {bib-example.tex}
\end{minipage}
\begin{minipage}{0.35\linewidth}
\includegraphics[width=1.4\textwidth,clip,trim=1.8in 5in 1.6in 1in]{bib-example.pdf}
\end{minipage}
\end{itemize}
\end{frame}

%%%%%%%%%%%%%%%%%%%%%%%%%%%%%%%%%%%%%%%%%%%%%%%%%%%%%%%%%%%%%%%%%%%%%%%%%%%%%%%
%%%%%%%%%%%%%%%%%%%%%%%%%%%%%%%%%%%%%%%%%%%%%%%%%%%%%%%%%%%%%%%%%%%%%%%%%%%%%%%
%%%%%%%%%%%%%%%%%%%%%%%%%%%%%%%%%%%%%%%%%%%%%%%%%%%%%%%%%%%%%%%%%%%%%%%%%%%%%%%
\subsection{Exercici}
\begin{frame}[fragile]{Exercici: Pose-m'ho tot junt}

Afegiu una imatge i bibliografia a l'article de l'exercici anterior.

\begin{enumerate}
\item Baixeu-vos aquests fitxers d'exemple (o els vostres propis).

\begin{center}
\fbox{\href{\fileuri/figures/gus_gran.png?dl=1}{Clica per baixar la imatge d'exemple}}

\fbox{\href{\fileuri/bib-exercise.bib?dl=1}{Clica per descarregar fitxer BIB d'exemple}}
\end{center}

\item Pujeu-los a Overleaf.

\end{enumerate}
\end{frame}

%%%%%%%%%%%%%%%%%%%%%%%%%%%%%%%%%%%%%%%%%%%%%%%%%%%%%%%%%%%%%%%%%%%%%%%%%%%%%%%
%%%%%%%%%%%%%%%%%%%%%%%%%%%%%%%%%%%%%%%%%%%%%%%%%%%%%%%%%%%%%%%%%%%%%%%%%%%%%%%
%%%%%%%%%%%%%%%%%%%%%%%%%%%%%%%%%%%%%%%%%%%%%%%%%%%%%%%%%%%%%%%%%%%%%%%%%%%%%%%
\section{Paquets lingüística}
\subsection{linguex}
\begin{frame}[fragile]{linguex: exemples}
\begin{itemize}
\item Les 3 ordres bàsiques de linguex són \cmdbs{ex.}, \cmdbs{a.} i \cmdbs{b.}.
\item \cmdbs{ex.} inicia l'exemple, i cal una línia en blanc per acabar-lo.
\begin{exampletwouptiny2}
\ex. Exemple

\ex. Primer nivell de l'exemple
\a. Segon nivell de l'exemple
\b. Seguim al segon nivell

Com hem vist a \LLast i \Last \dots

En canvi a \Next veiem \dots

\ex.
\a. podem deixar buit
\b. el primer nivell
\c. si volem ser ordenats
\d. podem fer servir
\e. tot l'abecedari
\b. tot i que no cal

\end{exampletwouptiny2}
\item Les ordres \cmdbs{c.}, \cmdbs{d.}, etc.~són còpies de \cmdbs{b.}.
\end{itemize}
\end{frame}

%%%%%%%%%%%%%%%%%%%%%%%%%%%%%%%%%%%%%%%%%%%%%%%%%%%%%%%%%%%%%%%%%%%%%%%%%%%%%%%
%%%%%%%%%%%%%%%%%%%%%%%%%%%%%%%%%%%%%%%%%%%%%%%%%%%%%%%%%%%%%%%%%%%%%%%%%%%%%%%
%%%%%%%%%%%%%%%%%%%%%%%%%%%%%%%%%%%%%%%%%%%%%%%%%%%%%%%%%%%%%%%%%%%%%%%%%%%%%%%
\begin{frame}[fragile]{linguex: exemples 2}
\begin{itemize}
\item Per acabar un sol nivell s'utilitza \cmdbs{z.}
\begin{exampletwouptiny2}
\ex. Animals
\a. Gats
\a. miau
\z.
\b. Gossos
\a. bup
\z.
\b. Fures

\end{exampletwouptiny2}
\end{itemize}

\end{frame}

%%%%%%%%%%%%%%%%%%%%%%%%%%%%%%%%%%%%%%%%%%%%%%%%%%%%%%%%%%%%%%%%%%%%%%%%%%%%%%%
%%%%%%%%%%%%%%%%%%%%%%%%%%%%%%%%%%%%%%%%%%%%%%%%%%%%%%%%%%%%%%%%%%%%%%%%%%%%%%%
%%%%%%%%%%%%%%%%%%%%%%%%%%%%%%%%%%%%%%%%%%%%%%%%%%%%%%%%%%%%%%%%%%%%%%%%%%%%%%%
\begin{frame}[fragile]{linguex: gramaticalitat}
\begin{itemize}
    \item Linguex admet els símbols *, ?, \# i \% per judicis de gramaticalitat (els posa al davant)\footnote{recordeu que \# i \% s'escriuen amb una \bs~al davant}.
\begin{exampletwouptiny2}
\ex.
\a. Exemple ben format
\b. * Agramatical frase?
\b. ** i molt agramatical molt
\b. \# El formatge l'hi he posat
\c. ? Hi han maduixes

\end{exampletwouptiny2}
\end{itemize}

\end{frame}

%%%%%%%%%%%%%%%%%%%%%%%%%%%%%%%%%%%%%%%%%%%%%%%%%%%%%%%%%%%%%%%%%%%%%%%%%%%%%%%
%%%%%%%%%%%%%%%%%%%%%%%%%%%%%%%%%%%%%%%%%%%%%%%%%%%%%%%%%%%%%%%%%%%%%%%%%%%%%%%
%%%%%%%%%%%%%%%%%%%%%%%%%%%%%%%%%%%%%%%%%%%%%%%%%%%%%%%%%%%%%%%%%%%%%%%%%%%%%%%
\begin{frame}[fragile]{linguex: glosses}
\begin{itemize}
\item Linguex és molt últil per morfologia.
\item Per glossar un exemple afegim `g' després de l'última lletra de l'ordre i linguex s'encarrega d'alinear les paraules (cal \keystrokebftt{\bs\bs}  per marcar el final de línia).
\begin{exampletwouptiny2}
\ex.\a. No gloss
\bg. This is a first gloss\\
Dies ist eine erste Glosse\\

\exg.
Dies ist nicht die erste Glosse\\
This is not the first gloss\\

\end{exampletwouptiny2}
\item Podem afegir una tercera línia de traducció amb \cmdbs{glt} (gloss translation).
\begin{exampletwouptiny2}
\exg.
Martin-ek Diego-\zero{} ikusi du \\
Martin-ERG Diego-ABS vist ha \\
\glt `En Martin ha vist en Diego'

\end{exampletwouptiny2}
\end{itemize}

\end{frame}

%%%%%%%%%%%%%%%%%%%%%%%%%%%%%%%%%%%%%%%%%%%%%%%%%%%%%%%%%%%%%%%%%%%%%%%%%%%%%%%
%%%%%%%%%%%%%%%%%%%%%%%%%%%%%%%%%%%%%%%%%%%%%%%%%%%%%%%%%%%%%%%%%%%%%%%%%%%%%%%
%%%%%%%%%%%%%%%%%%%%%%%%%%%%%%%%%%%%%%%%%%%%%%%%%%%%%%%%%%%%%%%%%%%%%%%%%%%%%%%
\begin{frame}[fragile]{linguex: glosses 2}
\begin{itemize}
\item Podem modificar les glosses perquè tinguin l'aspecte que vulguem.
\begin{exampletwouptiny}
\renewcommand{\eachwordone}{\itshape}
\renewcommand{\eachwordtwo}{\tiny}

\exg.
Martin-ek Diego-\zero{} ikusi du \\
Martin-ERG Diego-ABS vist ha \\
\glt `En Martin ha vist en Diego'


\end{exampletwouptiny}
\end{itemize}

\end{frame}

%%%%%%%%%%%%%%%%%%%%%%%%%%%%%%%%%%%%%%%%%%%%%%%%%%%%%%%%%%%%%%%%%%%%%%%%%%%%%%%
%%%%%%%%%%%%%%%%%%%%%%%%%%%%%%%%%%%%%%%%%%%%%%%%%%%%%%%%%%%%%%%%%%%%%%%%%%%%%%%
%%%%%%%%%%%%%%%%%%%%%%%%%%%%%%%%%%%%%%%%%%%%%%%%%%%%%%%%%%%%%%%%%%%%%%%%%%%%%%%
\begin{frame}[fragile]{linguex: claudators}
\begin{itemize}
\item Amb l'ordre \cmdbs{exi.} es pot fer servir notació amb claudators
\item Si l'exemple conté una glossa, cal saltar-se elements amb `\{\}'
\begin{MyMinted}
\exi.
\a.  [SC that [ST John$_i$  [Sv $t_i$ likes Mary ]]]
\bg. [SC dass [ST Peter$_i$ [Sv $t_i$ Mari liebt ]]] \\
{} that {} Peter {} {} Mary loves \\
\glt `that Peter loves Mary'
\end{MyMinted}
\exi.
\a.  [SC that [ST John$_i$  [Sv $t_i$ likes Mary ]]]
\bg. [SC dass [ST Peter$_i$ [Sv $t_i$ Mari liebt ]]] \\
{} that {} Peter {} {} Mary loves \\
\glt `that Peter loves Mary'

\end{itemize}

\end{frame}

%%%%%%%%%%%%%%%%%%%%%%%%%%%%%%%%%%%%%%%%%%%%%%%%%%%%%%%%%%%%%%%%%%%%%%%%%%%%%%%
%%%%%%%%%%%%%%%%%%%%%%%%%%%%%%%%%%%%%%%%%%%%%%%%%%%%%%%%%%%%%%%%%%%%%%%%%%%%%%%
%%%%%%%%%%%%%%%%%%%%%%%%%%%%%%%%%%%%%%%%%%%%%%%%%%%%%%%%%%%%%%%%%%%%%%%%%%%%%%%
\subsection{Extra}
\begin{frame}[fragile]{Extra: arbres}
\begin{itemize}
\item Hi ha dos paquets importants per fer arbres: forest i tikz-qtree.
\item Mireu els manuals per saber quin uns convé.
\begin{exampletwouptiny}
\ex. \begin{forest}
[SC[C][ST[T][SV[V][SN]]]]
\end{forest}

\end{exampletwouptiny}
\item \href{https://ling.auf.net/lingbuzz/003391}{Guia forest}
\end{itemize}

\end{frame}




%%%%%%%%%%%%%%%%%%%%%%%%%%%%%%%%%%%%%%%%%%%%%%%%%%%%%%%%%%%%%%%%%%%%%%%%%%%%%%%
%%%%%%%%%%%%%%%%%%%%%%%%%%%%%%%%%%%%%%%%%%%%%%%%%%%%%%%%%%%%%%%%%%%%%%%%%%%%%%%
%%%%%%%%%%%%%%%%%%%%%%%%%%%%%%%%%%%%%%%%%%%%%%%%%%%%%%%%%%%%%%%%%%%%%%%%%%%%%%%
\section{Què més?}
\begin{frame}{Continguts}
\begin{multicols}{2}
\tableofcontents[currentsection]
\end{multicols}
\end{frame}

%%%%%%%%%%%%%%%%%%%%%%%%%%%%%%%%%%%%%%%%%%%%%%%%%%%%%%%%%%%%%%%%%%%%%%%%%%%%%%%
%%%%%%%%%%%%%%%%%%%%%%%%%%%%%%%%%%%%%%%%%%%%%%%%%%%%%%%%%%%%%%%%%%%%%%%%%%%%%%%
%%%%%%%%%%%%%%%%%%%%%%%%%%%%%%%%%%%%%%%%%%%%%%%%%%%%%%%%%%%%%%%%%%%%%%%%%%%%%%%
\subsection{Altres coses interessants}
\begin{frame}[fragile]{Altres}
\begin{itemize}
\item L'ordre \cmdbs{tableofcontents} genera un índex automàticament a partir
dels títols de les seccions i subseccions.

\item Change the \cmdbs{documentclass} to
\mint{latex}!\documentclass{scrartcl}!
or
\mint{latex}!\documentclass[12pt]{IEEEtran}!

\item Define your own ordre for a complicated equation:
\begin{exampletwouptiny}
\newcommand{\rperf}{%
  \rho_{\text{perf}}}
$$
\rperf = {\bf c}'{\bf X} + \varepsilon
$$
\end{exampletwouptiny}
\end{itemize}
\end{frame}

%%%%%%%%%%%%%%%%%%%%%%%%%%%%%%%%%%%%%%%%%%%%%%%%%%%%%%%%%%%%%%%%%%%%%%%%%%%%%%%
%%%%%%%%%%%%%%%%%%%%%%%%%%%%%%%%%%%%%%%%%%%%%%%%%%%%%%%%%%%%%%%%%%%%%%%%%%%%%%%
%%%%%%%%%%%%%%%%%%%%%%%%%%%%%%%%%%%%%%%%%%%%%%%%%%%%%%%%%%%%%%%%%%%%%%%%%%%%%%%
\subsection{Altres paquets}
\begin{frame}{Altres paquets}
\begin{itemize}
\item \bftt{beamer}: per presentacions (com aquesta!)
\item \bftt{todonotes}: comenataris i coses pendents (TODO)
\item \bftt{tikz}: gràfics, una mica difícil.
\item \bftt{tikz-qtree}: arbres per sintaxi, relacions genètiques \dots
\item \bftt{pgfplots}: create graphs in \LaTeX
\item \bftt{langnames}: per noms d'idiomes de manera sistemàtica.
\end{itemize}
Vegeu \url{https://www.overleaf.com/latex/examples} i \url{http://texample.net}
per exemples.
\end{frame}

%%%%%%%%%%%%%%%%%%%%%%%%%%%%%%%%%%%%%%%%%%%%%%%%%%%%%%%%%%%%%%%%%%%%%%%%%%%%%%%
%%%%%%%%%%%%%%%%%%%%%%%%%%%%%%%%%%%%%%%%%%%%%%%%%%%%%%%%%%%%%%%%%%%%%%%%%%%%%%%
%%%%%%%%%%%%%%%%%%%%%%%%%%%%%%%%%%%%%%%%%%%%%%%%%%%%%%%%%%%%%%%%%%%%%%%%%%%%%%%
\subsection{Instalar \LaTeX{}}
\begin{frame}{Instalar \LaTeX}
\begin{itemize}
\item Per fer servir \LaTeX{} al vostre ordinador, us caldrà una \emph{distribució}:
    inclou el programa \bftt{latex} i un munt de paquets.
\begin{itemize}
\item Windows: \href{http://miktex.org/}{Mik\TeX} o \href{http://tug.org/texlive/}{\TeX Live}
\item Linux: \href{http://tug.org/texlive/}{\TeX Live}
\item Mac: \href{http://tug.org/mactex/}{Mac\TeX}
\end{itemize}
\item També us caldrà un editor amb suport per \LaTeX{}. A \url{http://en.wikipedia.org/wiki/Comparison_of_TeX_editors} hi teniu una llista exhaustiva.
\end{itemize}
\end{frame}

%%%%%%%%%%%%%%%%%%%%%%%%%%%%%%%%%%%%%%%%%%%%%%%%%%%%%%%%%%%%%%%%%%%%%%%%%%%%%%%
%%%%%%%%%%%%%%%%%%%%%%%%%%%%%%%%%%%%%%%%%%%%%%%%%%%%%%%%%%%%%%%%%%%%%%%%%%%%%%%
%%%%%%%%%%%%%%%%%%%%%%%%%%%%%%%%%%%%%%%%%%%%%%%%%%%%%%%%%%%%%%%%%%%%%%%%%%%%%%%
\subsection{Exemples comentats}
\begin{frame}{Exemples comentats}
De senzill a complex:
\begin{itemize}
    \item \href{https://www.overleaf.com/read/rgxbsfvqsgbw\#ba04b9}{Breu transcripció fonètica} --- amb \bftt{tipauni} o \bftt{tipa} (comentat).
    \item \href{https://www.overleaf.com/read/mxysfjcnnppm\#294a25}{Treball de transcripció fonètica} --- més extens que l'anterior, barreja de \bftt{tipa} i unicode.
    \item \href{https://www.overleaf.com/read/sxkcdfrpmbcb\#10abd3}{Treball morfologia basca} --- conté:
\begin{itemize}
    \item Un treball estructurat amb diversos fitxers, glosses personalitzades, taules amb format avançat i exemples de bibliografia tant amb biblatex com amb apacite.
    \item Una presentació amb \bftt{beamer}.
\end{itemize}
\end{itemize}

Finalment, us pot ser útil \href{https://www.overleaf.com/read/vsdvmbzywdrn\#d43fce}{aquesta plantilla senzilla}.

\end{frame}

%%%%%%%%%%%%%%%%%%%%%%%%%%%%%%%%%%%%%%%%%%%%%%%%%%%%%%%%%%%%%%%%%%%%%%%%%%%%%%%
%%%%%%%%%%%%%%%%%%%%%%%%%%%%%%%%%%%%%%%%%%%%%%%%%%%%%%%%%%%%%%%%%%%%%%%%%%%%%%%
%%%%%%%%%%%%%%%%%%%%%%%%%%%%%%%%%%%%%%%%%%%%%%%%%%%%%%%%%%%%%%%%%%%%%%%%%%%%%%%
\subsection{Referències en línia}
\begin{frame}{Referències en línia}
\begin{itemize}
\item \href{https://www.overleaf.com/learn}{The Overleaf Learn Wiki} --- tutorials i material de referència.
\item \href{http://en.wikibooks.org/wiki/LaTeX}{The \LaTeX{} Wikibook} --- més tutorials i material de referència.
\item \href{http://tex.stackexchange.com/}{\TeX{} Stack Exchange} --- preguntes i respostes
\item \href{http://www.latex-community.org/}{\LaTeX{} Community} --- fòrum
\item \href{http://ctan.org/}{Comprehensive \TeX{} Archive Network (CTAN)} --- documentació de paquets.
\item Qualsevol buscador us portarà a alguna d'aquestes pàgines.
\end{itemize}

Repositori del curs: \href{https://github.com/Pastells/curs-latex}{Pastells/curs-latex}
\end{frame}

\end{document}

\subsection{Tipografia}
\begin{frame}{Tipografia}
\begin{tabular}{lll}
& nom del símbol & s'usa per \dots \\\hline
\bftt{\bs} & backslash                 & ordres, tables \\
\bftt{\{}  & open brace                & ordres \\
\bftt{\}}  & close brace               & ordres \\
\bftt{\%}  & percent sign              & comments \\
\bftt{\#}  & hash (pound / sharp) sign & custom ordres \\
\bftt{\$}  & dollar sign               & equations \\
\bftt{\_}  & underscore                & equations (subscripts) \\
\bftt{\^}  & caret                     & equations (superscripts) \\
\bftt{\&}  & ampersand                 & tables \\
\bftt{\~}  & tilde                     & spacing \\
\end{tabular}
\end{frame}

% -- latex understands words, sentences i paragraphs

Words are separated by one or more spaces.  Paragraphs are separated by
one or more blank lines.  The output is not affected by adding extra
spaces or extra blank lines to the input file.

Double quotes are typed like this: ``quoted text''.
Single quotes are typed like this: `single-quoted text'.

Emphasized text is typed like this: \emph{this is emphasized}.
Bold       text is typed like this: \textbf{this is bold}.

-- Adding structure to your document

\section{Hello}

\subsection{World}

\subsection{Foo}

\subsubsection*{Stuff} % star form

\subsubsection*{Results}

-- Labels i cross-references

\label{sec:intro}
\label{sec:method}
\ref{sec:method}

--> maybe introduce the prettyref package here.

-- Mathematics

Inline mathematics: $x + y < 7$.

'Displayed' mathematics:
\begin{equation}
\end{equation}

\begin{equation*}
\end{equation*}

\begin{align}
\end{align}

-- Figures

- Need the graphicx package.

- here we can start introducing options

\includegraphics[width=\textwidth]{}

- where do you find out about these options? --> link to the Wikibook

-- Floating Figures

\begin{figure}
\includegraphics{...}
\caption{\label{}Here is a caption.}
\end{figure}

-- Tables

- not the nicest part of LaTeX

\usepackage{tabularx}

\begin{tabular}{llr}
Item & Quantity & Price (\$) & Amount
Widget & 1 &
\end{tabular}

Bonus points: check out the fp package i the spreadtab package.

-- Document Classes

a .cls file

article

some journal templates come with one

-- Bibliographies



-- For Typesetting Geeks

- dashes: -, --, ---

- ellipsis.

- controlling spaces: ~, \ , \,, \@

- spacing after periods (et al., etc.)

- Nested quotation marks: ``\,`
\vskip 2ex
\item Use the \emph{star form} to display an equation without a number.
\begin{exampletwouptiny}
\begin{equation*}
F(x) = \int_{a}^{x}{f(t) dt}
\end{equation*}
\end{exampletwouptiny}

\begin{itemize}
\item \bftt{equation} i \bftt{equation*} are called \emph{environments}.
\begin{itemize}
  \item The \cmdbs{begin} i \cmdbs{end} ordres define the environment.
  \item The \cmd{\$} also starts i ends an environment.
  \item Some ordres are defined only within certain environments.
  \item Some ordres behave differently in different environments.
\end{itemize}
\end{itemize}
\end{block}
\begin{center}
\fbox{\href{http://ctan.org/}{The Comprehensive \TeX Archive Network (CTAN)}}
\end{center}
