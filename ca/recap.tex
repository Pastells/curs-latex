\input{preamble.tex}
\date{}

\subtitle{Repàs}

\begin{document}

%%%%%%%%%%%%%%%%%%%%%%%%%%%%%%%%%%%%%%%%%%%%%%%%%%%%%%%%%%%%%%%%%%%%%%%%%%%%%%%
%%%%%%%%%%%%%%%%%%%%%%%%%%%%%%%%%%%%%%%%%%%%%%%%%%%%%%%%%%%%%%%%%%%%%%%%%%%%%%%
%%%%%%%%%%%%%%%%%%%%%%%%%%%%%%%%%%%%%%%%%%%%%%%%%%%%%%%%%%%%%%%%%%%%%%%%%%%%%%%
\begin{frame}
\titlepage
\end{frame}

%%%%%%%%%%%%%%%%%%%%%%%%%%%%%%%%%%%%%%%%%%%%%%%%%%%%%%%%%%%%%%%%%%%%%%%%%%%%%%%
%%%%%%%%%%%%%%%%%%%%%%%%%%%%%%%%%%%%%%%%%%%%%%%%%%%%%%%%%%%%%%%%%%%%%%%%%%%%%%%
%%%%%%%%%%%%%%%%%%%%%%%%%%%%%%%%%%%%%%%%%%%%%%%%%%%%%%%%%%%%%%%%%%%%%%%%%%%%%%%
\section{Repassem}

%%%%%%%%%%%%%%%%%%%%%%%%%%%%%%%%%%%%%%%%%%%%%%%%%%%%%%%%%%%%%%%%%%%%%%%%%%%%%%%
%%%%%%%%%%%%%%%%%%%%%%%%%%%%%%%%%%%%%%%%%%%%%%%%%%%%%%%%%%%%%%%%%%%%%%%%%%%%%%%
%%%%%%%%%%%%%%%%%%%%%%%%%%%%%%%%%%%%%%%%%%%%%%%%%%%%%%%%%%%%%%%%%%%%%%%%%%%%%%%
\begin{frame}[fragile]{Repassem}
\begin{itemize}
\item Escriviu el document a \texttt{text pla} amb \cmd{ordres} que
descriuen la seva estructura i significat.
\item Usa ordres per descriure `el contingut', no `el format'.
\item El programa \texttt{latex} processa el text i les ordres per a produir un
document amb un format bonic.
\end{itemize}
\vskip 2ex
\begin{center}
\begin{minted}[frame=single]{latex}
La \textit{fonologia} estudia els \textbf{fonemes}.
\end{minted}
\vskip 2ex
\tikz\node[single arrow,fill=gray,font=\ttfamily\bfseries,%
  rotate=270,xshift=-1em]{latex};
\vskip 2ex
\fbox{La \textit{fonologia} estudia els \textbf{fonemes}.}
\end{center}
\end{frame}

%%%%%%%%%%%%%%%%%%%%%%%%%%%%%%%%%%%%%%%%%%%%%%%%%%%%%%%%%%%%%%%%%%%%%%%%%%%%%%%
%%%%%%%%%%%%%%%%%%%%%%%%%%%%%%%%%%%%%%%%%%%%%%%%%%%%%%%%%%%%%%%%%%%%%%%%%%%%%%%
%%%%%%%%%%%%%%%%%%%%%%%%%%%%%%%%%%%%%%%%%%%%%%%%%%%%%%%%%%%%%%%%%%%%%%%%%%%%%%%
\begin{frame}[fragile]{Repassem: ordres i arguments}
\begin{itemize}
\item Les ordres comencen amb una \emph{barra inversa} \keystrokebftt{\bs}.
\item Algunes ordres admeten un \emph{argument} entre claus \keystrokebftt{\{}
\keystrokebftt{\}}.
\item Altres també admeten \emph{arguments opcionals} entre claudàtors \keystrokebftt{[} \keystrokebftt{]}.
\vskip 2ex
\begin{exampletwouptiny}
\includegraphics[
  width=0.5\textwidth]{khaleesi}

\includegraphics[
  angle=270]{gus}
\end{exampletwouptiny}
\end{itemize}

\end{frame}

%%%%%%%%%%%%%%%%%%%%%%%%%%%%%%%%%%%%%%%%%%%%%%%%%%%%%%%%%%%%%%%%%%%%%%%%%%%%%%%
%%%%%%%%%%%%%%%%%%%%%%%%%%%%%%%%%%%%%%%%%%%%%%%%%%%%%%%%%%%%%%%%%%%%%%%%%%%%%%%
%%%%%%%%%%%%%%%%%%%%%%%%%%%%%%%%%%%%%%%%%%%%%%%%%%%%%%%%%%%%%%%%%%%%%%%%%%%%%%%
\begin{frame}[fragile]{Repassem: entorns}
\begin{itemize}
\item Les ordres \cmdbs{begin} i \cmdbs{end} es fan servir per delimitar diferents entorns.
\vskip 2ex

\item Els entorns \bftt{itemize} i \bftt{enumerate} generen llistes de punts i numèriques. 
\begin{exampletwouptiny}
\begin{itemize} % per punts 
    \item Galetes
    \item Iogurt
\end{itemize}

\begin{enumerate} % per nombres
    \item Galetes
    \item Iogurt
\end{enumerate}
\end{exampletwouptiny}
\item No és necessari, però és bona pràctica augmentar el sagnat del text dins d'un entorn. 
\end{itemize}
\end{frame}

%%%%%%%%%%%%%%%%%%%%%%%%%%%%%%%%%%%%%%%%%%%%%%%%%%%%%%%%%%%%%%%%%%%%%%%%%%%%%%%
%%%%%%%%%%%%%%%%%%%%%%%%%%%%%%%%%%%%%%%%%%%%%%%%%%%%%%%%%%%%%%%%%%%%%%%%%%%%%%%
%%%%%%%%%%%%%%%%%%%%%%%%%%%%%%%%%%%%%%%%%%%%%%%%%%%%%%%%%%%%%%%%%%%%%%%%%%%%%%%
\begin{frame}[fragile]{Repassem: Matemàtiques}
\begin{itemize}
\item Els símbols de dòlar \keystrokebftt{\$} marquen matemàtiques al text.
\begin{exampletwouptiny}
% meh:
Siguin a i b nombres enters 
positius, i sigui c > a - b + 1

% millor:
Siguin $a$ i $b$ nombres enters 
positius, i sigui $c > a - b + 1$.
\end{exampletwouptiny}
\item Utilitza sempre els símbols de dòlar amb parelles: un per començar les matemàtiques, i un per acabar-les.
\end{itemize}
\end{frame}

%%%%%%%%%%%%%%%%%%%%%%%%%%%%%%%%%%%%%%%%%%%%%%%%%%%%%%%%%%%%%%%%%%%%%%%%%%%%%%%
%%%%%%%%%%%%%%%%%%%%%%%%%%%%%%%%%%%%%%%%%%%%%%%%%%%%%%%%%%%%%%%%%%%%%%%%%%%%%%%
%%%%%%%%%%%%%%%%%%%%%%%%%%%%%%%%%%%%%%%%%%%%%%%%%%%%%%%%%%%%%%%%%%%%%%%%%%%%%%%
\begin{frame}[fragile]{Repassem: Estructura Documents}
\begin{itemize}{\small
\item Comença amb \cmdbs{documentclass} --- quin tipus de document.
\item Metadades (\cmdbs{title} i \cmdbs{author}) i paquets en el preàmbul. 
\item Contingut entre \cmdbegin{document} i \cmdend{document}.
\item L'ordre \cmdbs{maketitle} crea el títol; \cmdbs{section} crea seccions numerades.
}\end{itemize}
\begin{minipage}{0.55\linewidth}
\inputminted[fontsize=\scriptsize,frame=single,resetmargins]{latex}%
  {recap-structure.tex}
\end{minipage}
\begin{minipage}{0.35\linewidth}
% trim: l b r t
\includegraphics[width=\textwidth,clip,trim=1.5in 7in 3in 2in]{recap-structure.pdf}
\end{minipage}
\end{frame}

%%%%%%%%%%%%%%%%%%%%%%%%%%%%%%%%%%%%%%%%%%%%%%%%%%%%%%%%%%%%%%%%%%%%%%%%%%%%%%%
%%%%%%%%%%%%%%%%%%%%%%%%%%%%%%%%%%%%%%%%%%%%%%%%%%%%%%%%%%%%%%%%%%%%%%%%%%%%%%%
%%%%%%%%%%%%%%%%%%%%%%%%%%%%%%%%%%%%%%%%%%%%%%%%%%%%%%%%%%%%%%%%%%%%%%%%%%%%%%%
\begin{frame}[fragile]{Repassem: Exercici}

\begin{enumerate}
\item Aquí teniu un text per un article curt:\footnote{Basat en \url{http://www.cgd.ucar.edu/cms/agu/scientific_talk.html}}
\begin{center}
\fbox{\href{\wlnewdoc{recap-exercise.tex}}{%
Cliqueu per obrir l'exercici a \wllogo{}}}
\end{center}
\vskip 2ex
\item Afegiu ordre de \LaTeX{} perquè tingui el mateix aspecte que el següent:
\begin{center}
\fbox{\href{\fileuri/recap-exercise-solution.pdf} amb una barra inversa (\cmdbs{\%}).
\end{itemize}
\end{block}
\end{frame}

\end{document}
