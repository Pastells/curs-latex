\input{preamble.tex}
\date{17 de maig de 2024}

\subtitle{Tercera part}

\newcommand{\mar}[1]{\todo[color=green!40]{#1}}
\newcommand{\andreu}[1]{\todo[color=purple!40]{#1}}

\begin{document}

%%%%%%%%%%%%%%%%%%%%%%%%%%%%%%%%%%%%%%%%%%%%%%%%%%%%%%%%%%%%%%%%%%%%%%%%%%%%%%%
%%%%%%%%%%%%%%%%%%%%%%%%%%%%%%%%%%%%%%%%%%%%%%%%%%%%%%%%%%%%%%%%%%%%%%%%%%%%%%%
%%%%%%%%%%%%%%%%%%%%%%%%%%%%%%%%%%%%%%%%%%%%%%%%%%%%%%%%%%%%%%%%%%%%%%%%%%%%%%%
\begin{frame}
\titlepage{}
\end{frame}

%%%%%%%%%%%%%%%%%%%%%%%%%%%%%%%%%%%%%%%%%%%%%%%%%%%%%%%%%%%%%%%%%%%%%%%%%%%%%%%
%%%%%%%%%%%%%%%%%%%%%%%%%%%%%%%%%%%%%%%%%%%%%%%%%%%%%%%%%%%%%%%%%%%%%%%%%%%%%%%
%%%%%%%%%%%%%%%%%%%%%%%%%%%%%%%%%%%%%%%%%%%%%%%%%%%%%%%%%%%%%%%%%%%%%%%%%%%%%%%
\begin{frame}[fragile]{Repassem: Dubtes classe anterior}
\begin{itemize}
    \item \textbf{Com es referencia una secció pel seu nom?}\\
    El paquet \bftt{hyperref} també carrega l'ordre \cmdbs{nameref}.
\begin{exampletwouptiny}
\nameref{sec:exemples_ling}
\end{exampletwouptiny}
\end{itemize}
\end{frame}

%%%%%%%%%%%%%%%%%%%%%%%%%%%%%%%%%%%%%%%%%%%%%%%%%%%%%%%%%%%%%%%%%%%%%%%%%%%%%%%
%%%%%%%%%%%%%%%%%%%%%%%%%%%%%%%%%%%%%%%%%%%%%%%%%%%%%%%%%%%%%%%%%%%%%%%%%%%%%%%
%%%%%%%%%%%%%%%%%%%%%%%%%%%%%%%%%%%%%%%%%%%%%%%%%%%%%%%%%%%%%%%%%%%%%%%%%%%%%%%
\section{Exemples lingüístics}\label{sec:exemples_ling}
\begin{frame}{Continguts}
\begin{multicols}{2}
\tableofcontents[currentsection]
\end{multicols}
\end{frame}


%%%%%%%%%%%%%%%%%%%%%%%%%%%%%%%%%%%%%%%%%%%%%%%%%%%%%%%%%%%%%%%%%%%%%%%%%%%%%%%
%%%%%%%%%%%%%%%%%%%%%%%%%%%%%%%%%%%%%%%%%%%%%%%%%%%%%%%%%%%%%%%%%%%%%%%%%%%%%%%
%%%%%%%%%%%%%%%%%%%%%%%%%%%%%%%%%%%%%%%%%%%%%%%%%%%%%%%%%%%%%%%%%%%%%%%%%%%%%%%
\subsection{linguex}
\begin{frame}[fragile]{linguex: exemples}
\begin{itemize}
\item Les 3 ordres bàsiques de linguex són \cmdbs{ex.}, \cmdbs{a.} i \cmdbs{b.}.
\item \cmdbs{ex.} inicia l'exemple, i \textbf{cal una línia en blanc} per acabar-lo.
\begin{exampletwouptiny2}
\ex. Exemple

\ex. Primer nivell de l'exemple
\a. Segon nivell de l'exemple
\b. Seguim al segon nivell

\ex.
\a. podem deixar buit
\b. el primer nivell
\c. si volem ser ordenats
\d. podem fer servir
\e. tot l'abecedari
\b. tot i que no cal

\end{exampletwouptiny2}
\item Les ordres \cmdbs{c.}, \cmdbs{d.}, etc.~són còpies de \cmdbs{b.}.
\item \bftt{linguex} és senzill, però, tot i poder-se modificar, és limitat. Hi ha paquets més potents però amb una sintaxi més complexa. \href{https://www.jostellings.com/numbex.html}{Vegeu aquesta breu explicació}.
\end{itemize}
\end{frame}

%%%%%%%%%%%%%%%%%%%%%%%%%%%%%%%%%%%%%%%%%%%%%%%%%%%%%%%%%%%%%%%%%%%%%%%%%%%%%%%
%%%%%%%%%%%%%%%%%%%%%%%%%%%%%%%%%%%%%%%%%%%%%%%%%%%%%%%%%%%%%%%%%%%%%%%%%%%%%%%
%%%%%%%%%%%%%%%%%%%%%%%%%%%%%%%%%%%%%%%%%%%%%%%%%%%%%%%%%%%%%%%%%%%%%%%%%%%%%%%
\begin{frame}[fragile]{linguex: referències}
\begin{itemize}
\item Podem fer servir \cmdbs{label} i \cmdbs{ref}.
\item També podem referir-nos als exemples més propers amb \cmdbs{Next}, \cmdbs{Last}, \cmdbs{NNext} i \cmdbs{LLast}, fins i tot si no tenen una etiqueta.
\end{itemize}
\begin{exampletwouptiny2}
A l'exemple~\ref{ex:1} hi ha faltes.
\ex. Ejenple
\label{ex:1}

\ex. Primer nivell de l'exemple
\a. Segon nivell de l'exemple

Com hem vist a \LLast i \Last \dots

En canvi a \Next i \NNext veiem \dots

\ex.
\a. podem deixar buit
\b. el primer nivell

\ex.
\label{ex:2}
\a. podem citar subexemples
\label{ex:2a}
\b. posant label a sota
\label{ex:2b}

Exemple \ref{ex:2} amb subexemples
\ref{ex:2a} i \ref{ex:2b}.

\end{exampletwouptiny2}
\end{frame}

%%%%%%%%%%%%%%%%%%%%%%%%%%%%%%%%%%%%%%%%%%%%%%%%%%%%%%%%%%%%%%%%%%%%%%%%%%%%%%%
%%%%%%%%%%%%%%%%%%%%%%%%%%%%%%%%%%%%%%%%%%%%%%%%%%%%%%%%%%%%%%%%%%%%%%%%%%%%%%%
%%%%%%%%%%%%%%%%%%%%%%%%%%%%%%%%%%%%%%%%%%%%%%%%%%%%%%%%%%%%%%%%%%%%%%%%%%%%%%%
\begin{frame}[fragile]{linguex: exemples 2}
\begin{itemize}
\item Per acabar un sol nivell s'utilitza \cmdbs{z.}
\begin{exampletwouptiny2}
\ex. Animals
\a. Gats
\a. miau
\z.
\b. Gossos
\a. bup
\z.
\b. Fures

\end{exampletwouptiny2}
\end{itemize}

\end{frame}

%%%%%%%%%%%%%%%%%%%%%%%%%%%%%%%%%%%%%%%%%%%%%%%%%%%%%%%%%%%%%%%%%%%%%%%%%%%%%%%
%%%%%%%%%%%%%%%%%%%%%%%%%%%%%%%%%%%%%%%%%%%%%%%%%%%%%%%%%%%%%%%%%%%%%%%%%%%%%%%
%%%%%%%%%%%%%%%%%%%%%%%%%%%%%%%%%%%%%%%%%%%%%%%%%%%%%%%%%%%%%%%%%%%%%%%%%%%%%%%
\begin{frame}[fragile]{linguex: gramaticalitat}
\begin{itemize}
    \item Linguex admet els símbols *, ?, \# i \% per judicis de gramaticalitat (els posa al davant)\footnote{recordeu que \# i \% s'escriuen amb una \keystroke{\textbackslash} davant}.
\begin{exampletwouptiny2}
\ex.
\a. Exemple ben format
\b. * Agramatical frase?
\b. ** i molt agramatical molt
\b. \# El formatge l'hi he posat
\c. ? Hi han maduixes

\end{exampletwouptiny2}
\end{itemize}

\end{frame}

%%%%%%%%%%%%%%%%%%%%%%%%%%%%%%%%%%%%%%%%%%%%%%%%%%%%%%%%%%%%%%%%%%%%%%%%%%%%%%%
%%%%%%%%%%%%%%%%%%%%%%%%%%%%%%%%%%%%%%%%%%%%%%%%%%%%%%%%%%%%%%%%%%%%%%%%%%%%%%%
%%%%%%%%%%%%%%%%%%%%%%%%%%%%%%%%%%%%%%%%%%%%%%%%%%%%%%%%%%%%%%%%%%%%%%%%%%%%%%%
\begin{frame}[fragile]{linguex: glosses}
\begin{itemize}
\item Linguex és molt últil per morfologia.
\item Per glossar un exemple afegim `g' després de l'última lletra de l'ordre i linguex s'encarrega d'alinear les paraules (cal \keystrokebftt{\bs\bs}  per marcar el final de línia).
\begin{exampletwouptiny2}
\ex.\a. No gloss
\bg. This is a first gloss\\
Dies ist eine erste Glosse\\

\exg.
Dies ist nicht die erste Glosse\\
This is not the first gloss\\

\end{exampletwouptiny2}
\item Podem afegir una tercera línia de traducció amb \cmdbs{glt} (gloss translation).
\begin{exampletwouptiny2}
\exg.
Martin-ek Diego-\zero{} ikusi du \\
Martin-ERG Diego-ABS vist ha \\
\glt `En Martin ha vist en Diego'

\end{exampletwouptiny2}
\end{itemize}

\end{frame}

%%%%%%%%%%%%%%%%%%%%%%%%%%%%%%%%%%%%%%%%%%%%%%%%%%%%%%%%%%%%%%%%%%%%%%%%%%%%%%%
%%%%%%%%%%%%%%%%%%%%%%%%%%%%%%%%%%%%%%%%%%%%%%%%%%%%%%%%%%%%%%%%%%%%%%%%%%%%%%%
%%%%%%%%%%%%%%%%%%%%%%%%%%%%%%%%%%%%%%%%%%%%%%%%%%%%%%%%%%%%%%%%%%%%%%%%%%%%%%%
\begin{frame}[fragile]{linguex: glosses 2}
\begin{itemize}
\item Podem modificar les glosses perquè tinguin l'aspecte que vulguem.
\begin{exampletwouptiny}
\renewcommand{\eachwordone}{\itshape}
\renewcommand{\eachwordtwo}{\tiny}

\exg.
Martin-ek Diego-\zero{} ikusi du \\
Martin-ERG Diego-ABS vist ha \\
\glt `En Martin ha vist en Diego'


\end{exampletwouptiny}
\end{itemize}

\end{frame}

%%%%%%%%%%%%%%%%%%%%%%%%%%%%%%%%%%%%%%%%%%%%%%%%%%%%%%%%%%%%%%%%%%%%%%%%%%%%%%%
%%%%%%%%%%%%%%%%%%%%%%%%%%%%%%%%%%%%%%%%%%%%%%%%%%%%%%%%%%%%%%%%%%%%%%%%%%%%%%%
%%%%%%%%%%%%%%%%%%%%%%%%%%%%%%%%%%%%%%%%%%%%%%%%%%%%%%%%%%%%%%%%%%%%%%%%%%%%%%%
\begin{frame}[fragile]{linguex: claudators}
\begin{itemize}
\item Amb l'ordre \cmdbs{exi.} es pot fer servir notació amb claudators
\item Si l'exemple conté una glossa, cal saltar-se elements amb `\{\}'
\begin{MyMinted}
\exi.
\a.  [SC that [ST John$_i$  [Sv $t_i$ likes Mary ]]]
\bg. [SC dass [ST Peter$_i$ [Sv $t_i$ Mari liebt ]]] \\
{} that {} Peter {} {} Mary loves \\
\glt `that Peter loves Mary'
\end{MyMinted}
\exi.
\a.  [SC that [ST John$_i$  [Sv $t_i$ likes Mary ]]]
\bg. [SC dass [ST Peter$_i$ [Sv $t_i$ Mari liebt ]]] \\
{} that {} Peter {} {} Mary loves \\
\glt `that Peter loves Mary'

\end{itemize}

\end{frame}

%%%%%%%%%%%%%%%%%%%%%%%%%%%%%%%%%%%%%%%%%%%%%%%%%%%%%%%%%%%%%%%%%%%%%%%%%%%%%%%
%%%%%%%%%%%%%%%%%%%%%%%%%%%%%%%%%%%%%%%%%%%%%%%%%%%%%%%%%%%%%%%%%%%%%%%%%%%%%%%
%%%%%%%%%%%%%%%%%%%%%%%%%%%%%%%%%%%%%%%%%%%%%%%%%%%%%%%%%%%%%%%%%%%%%%%%%%%%%%%
\begin{frame}[fragile]{Extra: arbres}
\begin{itemize}
\item Hi ha dos paquets importants per fer arbres: \bftt{forest} i \bftt{tikz-qtree}.
\item Mireu els manuals per saber quin uns convé.
\begin{exampletwouptiny}
\ex. \begin{forest}
[SC[C][ST[T][SV[V][SN]]]]
\end{forest}

\end{exampletwouptiny}
\item \href{https://ling.auf.net/lingbuzz/003391}{Guia forest}
\end{itemize}

\end{frame}


%%%%%%%%%%%%%%%%%%%%%%%%%%%%%%%%%%%%%%%%%%%%%%%%%%%%%%%%%%%%%%%%%%%%%%%%%%%%%%%
%%%%%%%%%%%%%%%%%%%%%%%%%%%%%%%%%%%%%%%%%%%%%%%%%%%%%%%%%%%%%%%%%%%%%%%%%%%%%%%
%%%%%%%%%%%%%%%%%%%%%%%%%%%%%%%%%%%%%%%%%%%%%%%%%%%%%%%%%%%%%%%%%%%%%%%%%%%%%%%
\section{Transcripció fonètica}
\begin{frame}[fragile]{tipa}
\begin{itemize}
\item
\end{itemize}
\begin{exampletwouptiny2}
\end{exampletwouptiny2}
\end{frame}

%%%%%%%%%%%%%%%%%%%%%%%%%%%%%%%%%%%%%%%%%%%%%%%%%%%%%%%%%%%%%%%%%%%%%%%%%%%%%%%
%%%%%%%%%%%%%%%%%%%%%%%%%%%%%%%%%%%%%%%%%%%%%%%%%%%%%%%%%%%%%%%%%%%%%%%%%%%%%%%
%%%%%%%%%%%%%%%%%%%%%%%%%%%%%%%%%%%%%%%%%%%%%%%%%%%%%%%%%%%%%%%%%%%%%%%%%%%%%%%
\begin{frame}[fragile]{Extra: diagrama vocals}
\begin{itemize}
\item El paquet \bftt{vowel} permet crear diagrames de vocals.
\end{itemize}
\begin{minipage}{0.55\linewidth}
\inputminted[fontsize=\tiny,frame=single,resetmargins]{latex}%
  {ipa-vowel.tex}
\end{minipage}
\hfill
\begin{minipage}{0.4\linewidth}
\includegraphics[width=\textwidth]{ipa-vowel.png}
\end{minipage}
\end{frame}

%%%%%%%%%%%%%%%%%%%%%%%%%%%%%%%%%%%%%%%%%%%%%%%%%%%%%%%%%%%%%%%%%%%%%%%%%%%%%%%
%%%%%%%%%%%%%%%%%%%%%%%%%%%%%%%%%%%%%%%%%%%%%%%%%%%%%%%%%%%%%%%%%%%%%%%%%%%%%%%
%%%%%%%%%%%%%%%%%%%%%%%%%%%%%%%%%%%%%%%%%%%%%%%%%%%%%%%%%%%%%%%%%%%%%%%%%%%%%%%
\section{Anotacions amb \protect\bftt{todonotes}}
\begin{frame}{Continguts}
\begin{multicols}{2}
\tableofcontents[currentsection]
\end{multicols}
\end{frame}

%%%%%%%%%%%%%%%%%%%%%%%%%%%%%%%%%%%%%%%%%%%%%%%%%%%%%%%%%%%%%%%%%%%%%%%%%%%%%%%
%%%%%%%%%%%%%%%%%%%%%%%%%%%%%%%%%%%%%%%%%%%%%%%%%%%%%%%%%%%%%%%%%%%%%%%%%%%%%%%
%%%%%%%%%%%%%%%%%%%%%%%%%%%%%%%%%%%%%%%%%%%%%%%%%%%%%%%%%%%%%%%%%%%%%%%%%%%%%%%
\begin{frame}[fragile]{Anotacions amb \protect\bftt{todonotes}}
\begin{itemize}
\item L'ordre \cmdbs{todo} del paquet \bftt{todonotes} és útil per deixar notes
per a un mateix o per co"laboradors.
\begin{exampletwouptiny}
\todo{afegir resultats}
\todo[color=blue!20]{arreglar taula}
\end{exampletwouptiny}
\vskip 2ex
\item Fes-ho senzill: defineix les teves pròpies ordres amb \cmdbs{newcommand}
\begin{minted}[fontsize=\scriptsize,frame=single]{latex}
\newcommand{\mar}[1]{\todo[color=green!40]{#1}}
\newcommand{\andreu}[1]{\todo[color=purple!40]{#1}}
\end{minted}
Ja no cal escriure tant:
\begin{exampletwouptiny}
\mar{afegir resultats}
\andreu{arreglar taula}
\end{exampletwouptiny}
\end{itemize}
\end{frame}

%%%%%%%%%%%%%%%%%%%%%%%%%%%%%%%%%%%%%%%%%%%%%%%%%%%%%%%%%%%%%%%%%%%%%%%%%%%%%%%
%%%%%%%%%%%%%%%%%%%%%%%%%%%%%%%%%%%%%%%%%%%%%%%%%%%%%%%%%%%%%%%%%%%%%%%%%%%%%%%
%%%%%%%%%%%%%%%%%%%%%%%%%%%%%%%%%%%%%%%%%%%%%%%%%%%%%%%%%%%%%%%%%%%%%%%%%%%%%%%
\begin{frame}[fragile]{Anotacions amb \protect\bftt{todonotes}}
\begin{columns}
  \begin{column}{0.4\textwidth}
    \begin{itemize}
    \item Amb beamer només s'admeten les notes en línia,
        però les notes de marge són compatibles amb els documents normals.
    \item També es poden enumerar les notes amb l'ordre \cmdbs{listoftodos}.
    \end{itemize}
  \end{column}
  \begin{column}{0.6\textwidth}
    \includegraphics[width=\textwidth,page=1]{todonotes-example}
  \end{column}
\end{columns}
\end{frame}

%%%%%%%%%%%%%%%%%%%%%%%%%%%%%%%%%%%%%%%%%%%%%%%%%%%%%%%%%%%%%%%%%%%%%%%%%%%%%%%
%%%%%%%%%%%%%%%%%%%%%%%%%%%%%%%%%%%%%%%%%%%%%%%%%%%%%%%%%%%%%%%%%%%%%%%%%%%%%%%
%%%%%%%%%%%%%%%%%%%%%%%%%%%%%%%%%%%%%%%%%%%%%%%%%%%%%%%%%%%%%%%%%%%%%%%%%%%%%%%
\section{Presentacions amb \protect\bftt{beamer}}

\begin{frame}{Continguts}
\begin{multicols}{2}
\tableofcontents[currentsection]
\end{multicols}
\end{frame}

%%%%%%%%%%%%%%%%%%%%%%%%%%%%%%%%%%%%%%%%%%%%%%%%%%%%%%%%%%%%%%%%%%%%%%%%%%%%%%%
%%%%%%%%%%%%%%%%%%%%%%%%%%%%%%%%%%%%%%%%%%%%%%%%%%%%%%%%%%%%%%%%%%%%%%%%%%%%%%%
%%%%%%%%%%%%%%%%%%%%%%%%%%%%%%%%%%%%%%%%%%%%%%%%%%%%%%%%%%%%%%%%%%%%%%%%%%%%%%%
\begin{frame}[fragile]{Presentacions amb \protect\bftt{beamer}}
\begin{itemize}
\item Beamer és un paquet per fer presentacions (com aquesta).
\item Es fa servir amb el tipus de document \bftt{beamer}.
\item Use the \bftt{frame} environment to create slides.
\item L'entorn \bftt{frame} crea diapositives.
\end{itemize}
\begin{minipage}{0.55\linewidth}
\inputminted[fontsize=\scriptsize,frame=single,resetmargins]{latex}%
  {beamer-minimal.tex}
\end{minipage}
\begin{minipage}{0.35\linewidth}
% trim: l b r t
\includegraphics[width=\textwidth,clip,trim=1in 1in 1in 1in]{beamer-minimal.pdf}
\end{minipage}
\end{frame}

%%%%%%%%%%%%%%%%%%%%%%%%%%%%%%%%%%%%%%%%%%%%%%%%%%%%%%%%%%%%%%%%%%%%%%%%%%%%%%%
%%%%%%%%%%%%%%%%%%%%%%%%%%%%%%%%%%%%%%%%%%%%%%%%%%%%%%%%%%%%%%%%%%%%%%%%%%%%%%%
%%%%%%%%%%%%%%%%%%%%%%%%%%%%%%%%%%%%%%%%%%%%%%%%%%%%%%%%%%%%%%%%%%%%%%%%%%%%%%%
\begin{frame}[fragile]{Presentacions amb \protect\bftt{beamer}}

\begin{itemize}
\item Podeu partir de l'exemple i anar provant coses.
\end{itemize}
\vskip 2ex
\begin{center}
\fbox{\href{\wlnewdoc{beamer-minimal.tex}}{%
Click to open the example document in \wllogo{}}}
\end{center}
\end{frame}

%%%%%%%%%%%%%%%%%%%%%%%%%%%%%%%%%%%%%%%%%%%%%%%%%%%%%%%%%%%%%%%%%%%%%%%%%%%%%%%
%%%%%%%%%%%%%%%%%%%%%%%%%%%%%%%%%%%%%%%%%%%%%%%%%%%%%%%%%%%%%%%%%%%%%%%%%%%%%%%
%%%%%%%%%%%%%%%%%%%%%%%%%%%%%%%%%%%%%%%%%%%%%%%%%%%%%%%%%%%%%%%%%%%%%%%%%%%%%%%
\subsection{Marcs}
\begin{frame}[fragile]
\frametitle{Presentation amb \protect\bftt{beamer}: Marcs}
\begin{itemize}
\item Use \cmdbs{frametitle} to give the frame a title.
\item Then add content to the frame.
\item The source for this frame looks like:
\vskip 2ex
\inputminted[fontsize=\scriptsize,frame=single,resetmargins]{latex}%
  {beamer-frame.tex}
\end{itemize}
\end{frame}

%%%%%%%%%%%%%%%%%%%%%%%%%%%%%%%%%%%%%%%%%%%%%%%%%%%%%%%%%%%%%%%%%%%%%%%%%%%%%%%
%%%%%%%%%%%%%%%%%%%%%%%%%%%%%%%%%%%%%%%%%%%%%%%%%%%%%%%%%%%%%%%%%%%%%%%%%%%%%%%
%%%%%%%%%%%%%%%%%%%%%%%%%%%%%%%%%%%%%%%%%%%%%%%%%%%%%%%%%%%%%%%%%%%%%%%%%%%%%%%
\subsection{Seccions}
\begin{frame}[fragile]{Presentation amb \protect\bftt{beamer}: Seccions}
\begin{itemize}
\item You can use \cmdbs{section}s to group your \bftt{frame}s, i
\bftt{beamer} will use them to create an automatic outline.
\item To generate an outline, use the \cmdbs{tableofcontents} ordre. Here's
one for this presentation. The \bftt{currentsection} option highlights the current section.
\vskip 2ex
\begin{exampletwouptiny}
% comentaris

% per

% guanyar

% espai

\tableofcontents[currentsection]
\end{exampletwouptiny}
\end{itemize}
\end{frame}

%%%%%%%%%%%%%%%%%%%%%%%%%%%%%%%%%%%%%%%%%%%%%%%%%%%%%%%%%%%%%%%%%%%%%%%%%%%%%%%
%%%%%%%%%%%%%%%%%%%%%%%%%%%%%%%%%%%%%%%%%%%%%%%%%%%%%%%%%%%%%%%%%%%%%%%%%%%%%%%
%%%%%%%%%%%%%%%%%%%%%%%%%%%%%%%%%%%%%%%%%%%%%%%%%%%%%%%%%%%%%%%%%%%%%%%%%%%%%%%
\subsection{Columnes}
\begin{frame}[fragile]{Presentation amb \protect\bftt{beamer}: Multiple Columns}
\begin{columns}
\begin{column}{0.4\textwidth}
\begin{itemize}
\item Use the \bftt{columns} i \bftt{column} environments to break the slide
into columns.
\item The argument for each \bftt{column} determines its width.
\item See also the \bftt{multicol} package, which automatically breaks your
content into columns.
\end{itemize}
\end{column}
\begin{column}{0.6\textwidth}
\begin{minted}[fontsize=\scriptsize,frame=single]{latex}
\begin{columns}
  \begin{column}{0.4\textwidth}
    \begin{itemize}
    \item Use the columns ...
    \item The argument ...
    \item See also the ...
    \end{itemize}
  \end{column}
  \begin{column}{0.6\textwidth}
    % second column
  \end{column}
\end{columns}
\end{minted}
\end{column}
\end{columns}
\end{frame}

%%%%%%%%%%%%%%%%%%%%%%%%%%%%%%%%%%%%%%%%%%%%%%%%%%%%%%%%%%%%%%%%%%%%%%%%%%%%%%%
%%%%%%%%%%%%%%%%%%%%%%%%%%%%%%%%%%%%%%%%%%%%%%%%%%%%%%%%%%%%%%%%%%%%%%%%%%%%%%%
%%%%%%%%%%%%%%%%%%%%%%%%%%%%%%%%%%%%%%%%%%%%%%%%%%%%%%%%%%%%%%%%%%%%%%%%%%%%%%%
\begin{frame}[fragile]{Presentation amb \protect\bftt{beamer}: Destacar}
\begin{itemize}

\item Use \cmdbs{emph} or \cmdbs{alert} to highlight:
\vskip 1ex
\begin{exampletwouptiny}
I should \emph{emphasise} that
this is an \alert{important} point.
\end{exampletwouptiny}
\vskip 1ex

\item Or specify bold face or italics:
\vskip 1ex
\begin{exampletwouptiny}
Text in \textbf{bold face}.
Text in \textit{italics}.
\end{exampletwouptiny}
\vskip 1ex

\item Or specify a color (American spelling):
\vskip 1ex
\begin{exampletwouptiny}
It \textcolor{red}{stops}
and \textcolor{green}{starts}.
\end{exampletwouptiny}
\vskip 1ex
\item See \url{http://www.math.umbc.edu/~rouben/beamer/quickstart-Z-H-25.html}
for more colors \& custom colors.
\end{itemize}
\end{frame}

%%%%%%%%%%%%%%%%%%%%%%%%%%%%%%%%%%%%%%%%%%%%%%%%%%%%%%%%%%%%%%%%%%%%%%%%%%%%%%%
%%%%%%%%%%%%%%%%%%%%%%%%%%%%%%%%%%%%%%%%%%%%%%%%%%%%%%%%%%%%%%%%%%%%%%%%%%%%%%%
%%%%%%%%%%%%%%%%%%%%%%%%%%%%%%%%%%%%%%%%%%%%%%%%%%%%%%%%%%%%%%%%%%%%%%%%%%%%%%%
\subsection{Figures}
\begin{frame}[fragile]{Presentation amb \protect\bftt{beamer}: Figures}
\begin{itemize}
\item Use \cmdbs{includegraphics} from the \bftt{graphicx} package.
\item The \bftt{figure} environment centers by default, in \bftt{beamer}.
\vskip 2ex
\begin{exampletwouptiny}
\begin{figure}
\includegraphics[
  width=0.5\textwidth]{gus_gran}
\end{figure}
\end{exampletwouptiny}
\end{itemize}

\end{frame}

%%%%%%%%%%%%%%%%%%%%%%%%%%%%%%%%%%%%%%%%%%%%%%%%%%%%%%%%%%%%%%%%%%%%%%%%%%%%%%%
%%%%%%%%%%%%%%%%%%%%%%%%%%%%%%%%%%%%%%%%%%%%%%%%%%%%%%%%%%%%%%%%%%%%%%%%%%%%%%%
%%%%%%%%%%%%%%%%%%%%%%%%%%%%%%%%%%%%%%%%%%%%%%%%%%%%%%%%%%%%%%%%%%%%%%%%%%%%%%%
\begin{frame}[fragile]{Presentation amb \protect\bftt{beamer}: Blocks}
\begin{itemize}
\item A \bftt{block} environment makes a titled box.
\begin{exampletwouptiny}
\begin{block}{Interesting Fact}
This is important.
\end{block}

\begin{alertblock}{Cautionary Tale}
This is really important!
\end{alertblock}
\end{exampletwouptiny}

\item How exactly they look depends on the theme\dots
\end{itemize}
\end{frame}

%%%%%%%%%%%%%%%%%%%%%%%%%%%%%%%%%%%%%%%%%%%%%%%%%%%%%%%%%%%%%%%%%%%%%%%%%%%%%%%
%%%%%%%%%%%%%%%%%%%%%%%%%%%%%%%%%%%%%%%%%%%%%%%%%%%%%%%%%%%%%%%%%%%%%%%%%%%%%%%
%%%%%%%%%%%%%%%%%%%%%%%%%%%%%%%%%%%%%%%%%%%%%%%%%%%%%%%%%%%%%%%%%%%%%%%%%%%%%%%
\begin{frame}[fragile]
    \frametitle{Presentation amb \protect\bftt{beamer}: Themes}
\begin{itemize}
\item Customise the look of your presentation using themes.
\item See \url{http://deic.uab.es/~iblanes/beamer_gallery/index_by_theme.html}
for a large collection of themes.
\end{itemize}
\begin{minipage}{0.55\linewidth}
\inputminted[fontsize=\scriptsize,frame=single,resetmargins]{latex}%
  {beamer-theme.tex}
\end{minipage}
\begin{minipage}{0.35\linewidth}
% trim: l b r t
\includegraphics[width=\textwidth]{beamer-theme.pdf}
\end{minipage}
\end{frame}

%%%%%%%%%%%%%%%%%%%%%%%%%%%%%%%%%%%%%%%%%%%%%%%%%%%%%%%%%%%%%%%%%%%%%%%%%%%%%%%
%%%%%%%%%%%%%%%%%%%%%%%%%%%%%%%%%%%%%%%%%%%%%%%%%%%%%%%%%%%%%%%%%%%%%%%%%%%%%%%
%%%%%%%%%%%%%%%%%%%%%%%%%%%%%%%%%%%%%%%%%%%%%%%%%%%%%%%%%%%%%%%%%%%%%%%%%%%%%%%
\begin{frame}[fragile]{Presentation amb \protect\bftt{beamer}: Animation}
\begin{itemize}
\item A frame can generate multiple slides.
\item Use the \cmdbs{pause} ordre to show only part of a slide.
\vskip 2ex
\begin{exampletwouptinynoframe}
\begin{itemize}
\item Can you feel the
\pause \item anticipation?
\end{itemize}
\end{exampletwouptinynoframe}
\vskip 2ex
\item There many more clever ways of making animations in \bftt{beamer}; see
also the \cmdbs{only}, \cmdbs{alt}, i \cmdbs{uncover} ordres.
\end{itemize}
\end{frame}

%%%%%%%%%%%%%%%%%%%%%%%%%%%%%%%%%%%%%%%%%%%%%%%%%%%%%%%%%%%%%%%%%%%%%%%%%%%%%%%
%%%%%%%%%%%%%%%%%%%%%%%%%%%%%%%%%%%%%%%%%%%%%%%%%%%%%%%%%%%%%%%%%%%%%%%%%%%%%%%
%%%%%%%%%%%%%%%%%%%%%%%%%%%%%%%%%%%%%%%%%%%%%%%%%%%%%%%%%%%%%%%%%%%%%%%%%%%%%%%
\begin{frame}[fragile]{Presentation amb \protect\bftt{beamer}: Exercise}

Recreate Peter Norvig's excellent ``Gettysburg Powerpoint Presentation'' in \bftt{beamer}.\footnote{\url{http://norvig.com/Gettysburg}}

\begin{enumerate}
\item Open this exercise in \wllogo{}:
\begin{center}
\fbox{\href{\wlnewdoc{beamer-exercise.tex}}{%
Click to open this exercise in \wllogo{}}}
\end{center}
\vskip 2ex
\item Download this image to your computer i upload it to \wllogo{} via the
files menu.
\begin{center}
\fbox{\href{\fileuri/gettysburg_graph.png?dl=1}{Click to download image}}
\end{center}
\vskip 2ex
\item Add \LaTeX{} ordres to the text to make it look like this one:
\begin{center}
\fbox{\href{\fileuri/beamer-exercise-solution.pdf}{%
Click to open the model document}}
\end{center}
\end{enumerate}
\end{frame}

%%%%%%%%%%%%%%%%%%%%%%%%%%%%%%%%%%%%%%%%%%%%%%%%%%%%%%%%%%%%%%%%%%%%%%%%%%%%%%%
%%%%%%%%%%%%%%%%%%%%%%%%%%%%%%%%%%%%%%%%%%%%%%%%%%%%%%%%%%%%%%%%%%%%%%%%%%%%%%%
%%%%%%%%%%%%%%%%%%%%%%%%%%%%%%%%%%%%%%%%%%%%%%%%%%%%%%%%%%%%%%%%%%%%%%%%%%%%%%%
\section{Què més?}
\begin{frame}{Continguts}
\begin{multicols}{2}
\tableofcontents[currentsection]
\end{multicols}
\end{frame}

%%%%%%%%%%%%%%%%%%%%%%%%%%%%%%%%%%%%%%%%%%%%%%%%%%%%%%%%%%%%%%%%%%%%%%%%%%%%%%%
%%%%%%%%%%%%%%%%%%%%%%%%%%%%%%%%%%%%%%%%%%%%%%%%%%%%%%%%%%%%%%%%%%%%%%%%%%%%%%%
%%%%%%%%%%%%%%%%%%%%%%%%%%%%%%%%%%%%%%%%%%%%%%%%%%%%%%%%%%%%%%%%%%%%%%%%%%%%%%%
\subsection{Altres coses interessants}
\begin{frame}[fragile]{Altres}
\begin{itemize}
    \item \cmdbs{\textbackslash} per forçar canvi de línia i \cmdbs{newpage} per canvi de pàgina (\href{https://www.overleaf.com/learn/latex/Line_breaks_and_blank_spaces#Line_breaks}{més sobre espaiat}).
\item \cmdbs{footnote} per notes al peu de pàgina.
\item \cmdbs{input} permet incloure documents de \LaTeX{} dins d'altres documents.
    Útil per:
        \begin{itemize}{\footnotesize
        \item Seccions de documents
        \item Taules complexes
        \item Separar el preàmbul (i reutilitzar-lo)
        }\end{itemize}
        \footnotesize{Vegeu \textit{treball morfologia basca} als exemples finals.}
        \href{https://www.overleaf.com/learn/latex/Management_in_a_large_project}{Existeixen altres opcions per tractar amb documents grans}.
    \item Podem canviar la indentació dels paràgrafs (\href{https://www.overleaf.com/learn/latex/Articles/How_to_change_paragraph_spacing_in_LaTeX}{més sobre espaiat de paràgrafs}). \\
    \footnotesize{Vegeu \textit{treball de transcripció fonètica} als exemples finals.}
\begin{exampletiny}
%  Treu el sagnat
\setlength{\parindent}{0pt}
%  Passada la \tableofcontents afegeix espai en blanc
\setlength{\parskip}{\baselineskip}
\end{exampletiny}
\end{itemize}
\end{frame}

%%%%%%%%%%%%%%%%%%%%%%%%%%%%%%%%%%%%%%%%%%%%%%%%%%%%%%%%%%%%%%%%%%%%%%%%%%%%%%%
%%%%%%%%%%%%%%%%%%%%%%%%%%%%%%%%%%%%%%%%%%%%%%%%%%%%%%%%%%%%%%%%%%%%%%%%%%%%%%%
%%%%%%%%%%%%%%%%%%%%%%%%%%%%%%%%%%%%%%%%%%%%%%%%%%%%%%%%%%%%%%%%%%%%%%%%%%%%%%%
\subsection{Altres paquets}
\begin{frame}{Altres paquets}
\begin{itemize}
\item \bftt{tikz}: gràfics, una mica difícil.
\item \bftt{tikz-qtree}: arbres per sintaxi, relacions genètiques \dots
\item \bftt{pgfplots}: create graphs in \LaTeX{}
\item \bftt{langnames}: per noms d'idiomes de manera sistemàtica.
\end{itemize}
Vegeu \url{https://www.overleaf.com/latex/examples} i \url{http://texample.net}
per exemples.
\end{frame}

%%%%%%%%%%%%%%%%%%%%%%%%%%%%%%%%%%%%%%%%%%%%%%%%%%%%%%%%%%%%%%%%%%%%%%%%%%%%%%%
%%%%%%%%%%%%%%%%%%%%%%%%%%%%%%%%%%%%%%%%%%%%%%%%%%%%%%%%%%%%%%%%%%%%%%%%%%%%%%%
%%%%%%%%%%%%%%%%%%%%%%%%%%%%%%%%%%%%%%%%%%%%%%%%%%%%%%%%%%%%%%%%%%%%%%%%%%%%%%%
\subsection{Instalar \LaTeX{}}
\begin{frame}{Instalar \LaTeX}
\begin{itemize}
\item Per fer servir \LaTeX{} al vostre ordinador, us caldrà una \emph{distribució}:
    inclou el programa \bftt{latex} i un munt de paquets.
\begin{itemize}
\item Windows: \href{http://miktex.org/}{Mik\TeX} o \href{http://tug.org/texlive/}{\TeX Live}
\item Linux: \href{http://tug.org/texlive/}{\TeX Live}
\item Mac: \href{http://tug.org/mactex/}{Mac\TeX}
\end{itemize}
\item També us caldrà un editor amb suport per \LaTeX{}. A \url{http://en.wikipedia.org/wiki/Comparison_of_TeX_editors} hi teniu una llista exhaustiva.
\end{itemize}
\end{frame}

%%%%%%%%%%%%%%%%%%%%%%%%%%%%%%%%%%%%%%%%%%%%%%%%%%%%%%%%%%%%%%%%%%%%%%%%%%%%%%%
%%%%%%%%%%%%%%%%%%%%%%%%%%%%%%%%%%%%%%%%%%%%%%%%%%%%%%%%%%%%%%%%%%%%%%%%%%%%%%%
%%%%%%%%%%%%%%%%%%%%%%%%%%%%%%%%%%%%%%%%%%%%%%%%%%%%%%%%%%%%%%%%%%%%%%%%%%%%%%%
\subsection{Exemples comentats}\label{sec:exemples_comentats}
\begin{frame}{Exemples comentats}
De senzill a complex:
\begin{itemize}
    \item \href{https://www.overleaf.com/read/rgxbsfvqsgbw\#ba04b9}{Breu transcripció fonètica} --- amb \bftt{tipauni} o \bftt{tipa} (comentat).
    \item \href{https://www.overleaf.com/read/mxysfjcnnppm\#294a25}{Treball de transcripció fonètica} --- més extens que l'anterior, barreja de \bftt{tipa} i unicode.
    \item \href{https://www.overleaf.com/read/sxkcdfrpmbcb\#10abd3}{Treball morfologia basca} --- conté:
\begin{itemize}
    \item Un treball estructurat amb diversos fitxers, glosses personalitzades, taules amb format avançat i exemples de bibliografia tant amb biblatex com amb apacite.
    \item Una presentació amb \bftt{beamer}.
\end{itemize}
\end{itemize}

Finalment, us pot ser útil \href{https://www.overleaf.com/read/vsdvmbzywdrn\#d43fce}{aquesta plantilla senzilla}.

\end{frame}

%%%%%%%%%%%%%%%%%%%%%%%%%%%%%%%%%%%%%%%%%%%%%%%%%%%%%%%%%%%%%%%%%%%%%%%%%%%%%%%
%%%%%%%%%%%%%%%%%%%%%%%%%%%%%%%%%%%%%%%%%%%%%%%%%%%%%%%%%%%%%%%%%%%%%%%%%%%%%%%
%%%%%%%%%%%%%%%%%%%%%%%%%%%%%%%%%%%%%%%%%%%%%%%%%%%%%%%%%%%%%%%%%%%%%%%%%%%%%%%
\subsection{Referències en línia}
\begin{frame}{Referències en línia}
\begin{itemize}
\item \href{https://www.overleaf.com/learn}{The Overleaf Learn Wiki} --- tutorials i material de referència.
\item \href{http://en.wikibooks.org/wiki/LaTeX}{The \LaTeX{} Wikibook} --- més tutorials i material de referència.
\item \href{http://tex.stackexchange.com/}{\TeX{} Stack Exchange} --- preguntes i respostes
\item \href{http://www.latex-community.org/}{\LaTeX{} Community} --- fòrum
\item \href{http://ctan.org/}{Comprehensive \TeX{} Archive Network (CTAN)} --- documentació de paquets.
\item Qualsevol buscador us portarà a alguna d'aquestes pàgines.
\end{itemize}

Repositori del curs: \href{https://github.com/Pastells/curs-latex}{Pastells/curs-latex}
\end{frame}


%%%%%%%%%%%%%%%%%%%%%%%%%%%%%%%%%%%%%%%%%%%%%%%%%%%%%%%%%%%%%%%%%%%%%%%%%%%%%%%
%%%%%%%%%%%%%%%%%%%%%%%%%%%%%%%%%%%%%%%%%%%%%%%%%%%%%%%%%%%%%%%%%%%%%%%%%%%%%%%
%%%%%%%%%%%%%%%%%%%%%%%%%%%%%%%%%%%%%%%%%%%%%%%%%%%%%%%%%%%%%%%%%%%%%%%%%%%%%%%
\begin{frame}[standout]
\Huge Gràcies!
\end{frame}

\end{document}
