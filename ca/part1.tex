\input{preamble.tex}

\subtitle{Part 1: Fonaments}

\begin{document}

%%%%%%%%%%%%%%%%%%%%%%%%%%%%%%%%%%%%%%%%%%%%%%%%%%%%%%%%%%%%%%%%%%%%%%%%%%%%%%%
%%%%%%%%%%%%%%%%%%%%%%%%%%%%%%%%%%%%%%%%%%%%%%%%%%%%%%%%%%%%%%%%%%%%%%%%%%%%%%%
%%%%%%%%%%%%%%%%%%%%%%%%%%%%%%%%%%%%%%%%%%%%%%%%%%%%%%%%%%%%%%%%%%%%%%%%%%%%%%%
\begin{frame}
\titlepage
\end{frame}

%%%%%%%%%%%%%%%%%%%%%%%%%%%%%%%%%%%%%%%%%%%%%%%%%%%%%%%%%%%%%%%%%%%%%%%%%%%%%%%
%%%%%%%%%%%%%%%%%%%%%%%%%%%%%%%%%%%%%%%%%%%%%%%%%%%%%%%%%%%%%%%%%%%%%%%%%%%%%%%
%%%%%%%%%%%%%%%%%%%%%%%%%%%%%%%%%%%%%%%%%%%%%%%%%%%%%%%%%%%%%%%%%%%%%%%%%%%%%%%
\begin{frame}{Per què \LaTeX{}?}
\begin{itemize}
\item Fa documents amb una bona presentació
\begin{itemize}
\item Especialment per a matemàtiques i lingüística
\end{itemize}
%
\item Creat per científics, per a científics
\begin{itemize}
\item Una comunitat gran i activa
\end{itemize}
%
\item És potent --- podeu estendre-ho
\begin{itemize}
\item Paquets per a treballs, presentacions, tesis \dots
\end{itemize}
\end{itemize}
\end{frame}

%%%%%%%%%%%%%%%%%%%%%%%%%%%%%%%%%%%%%%%%%%%%%%%%%%%%%%%%%%%%%%%%%%%%%%%%%%%%%%%
%%%%%%%%%%%%%%%%%%%%%%%%%%%%%%%%%%%%%%%%%%%%%%%%%%%%%%%%%%%%%%%%%%%%%%%%%%%%%%%
%%%%%%%%%%%%%%%%%%%%%%%%%%%%%%%%%%%%%%%%%%%%%%%%%%%%%%%%%%%%%%%%%%%%%%%%%%%%%%%
\begin{frame}[fragile]{Com funciona?}
\begin{itemize}
\item Escriviu el document a \texttt{text pla} amb \cmd{ordres} que
descriuen la seva estructura i significat.
\item Usa ordres per descriure `el contingut', no `el format'.
\item El programa \texttt{latex} processa el text i les ordres per a produir un
document amb un format bonic.
\end{itemize}
\vskip 2ex
\begin{center}
\begin{minted}[frame=single]{latex}
La \textit{fonologia} estudia els \textbf{fonemes}.
\end{minted}
\vskip 2ex
\tikz\node[single arrow,fill=gray,font=\ttfamily\bfseries,%
  rotate=270,xshift=-1em]{latex};
\vskip 2ex
\fbox{La \textit{fonologia} estudia els \textbf{fonemes}.}
\end{center}
\end{frame}

%%%%%%%%%%%%%%%%%%%%%%%%%%%%%%%%%%%%%%%%%%%%%%%%%%%%%%%%%%%%%%%%%%%%%%%%%%%%%%%
%%%%%%%%%%%%%%%%%%%%%%%%%%%%%%%%%%%%%%%%%%%%%%%%%%%%%%%%%%%%%%%%%%%%%%%%%%%%%%%
%%%%%%%%%%%%%%%%%%%%%%%%%%%%%%%%%%%%%%%%%%%%%%%%%%%%%%%%%%%%%%%%%%%%%%%%%%%%%%%
\begin{frame}[fragile]{Exemples d'ordres i la seva sortida}
\begin{exampletwoup}
\begin{itemize}
\item Mel
\item Llet
\item Galetes
\end{itemize}
\end{exampletwoup}
\vskip 2ex
\begin{exampletwoup}
\begin{figure}
\includegraphics{gerbil}
\end{figure}
\end{exampletwoup}
\vskip 2ex
\begin{exampletwoup}
\begin{equation}
\alpha + \beta + 1
\end{equation}
\end{exampletwoup}

\tiny{Imatge: \href{https://pixabay.com/en/animal-apple-attractive-beautiful-1239390/}{CC0}}
\end{frame}


%%%%%%%%%%%%%%%%%%%%%%%%%%%%%%%%%%%%%%%%%%%%%%%%%%%%%%%%%%%%%%%%%%%%%%%%%%%%%%%
%%%%%%%%%%%%%%%%%%%%%%%%%%%%%%%%%%%%%%%%%%%%%%%%%%%%%%%%%%%%%%%%%%%%%%%%%%%%%%%
%%%%%%%%%%%%%%%%%%%%%%%%%%%%%%%%%%%%%%%%%%%%%%%%%%%%%%%%%%%%%%%%%%%%%%%%%%%%%%%
\section{Fonaments}

%%%%%%%%%%%%%%%%%%%%%%%%%%%%%%%%%%%%%%%%%%%%%%%%%%%%%%%%%%%%%%%%%%%%%%%%%%%%%%%
%%%%%%%%%%%%%%%%%%%%%%%%%%%%%%%%%%%%%%%%%%%%%%%%%%%%%%%%%%%%%%%%%%%%%%%%%%%%%%%
%%%%%%%%%%%%%%%%%%%%%%%%%%%%%%%%%%%%%%%%%%%%%%%%%%%%%%%%%%%%%%%%%%%%%%%%%%%%%%%
\subsection{Començant}
\begin{frame}[fragile]{Començant}
\begin{itemize}
\item Un document de \LaTeX{} mínim:
\inputminted[frame=single]{latex}{basics.tex}
\item Les ordres comencen amb una \emph{barra inversa} \keystrokebftt{\bs}.
\item Cada document comença amb una ordre \cmdbs{documentclass}.
\item L'\emph{argument} entre claus \keystrokebftt{\{} \keystrokebftt{\}} li diu a \LaTeX{} quin tipus de document estem creant: un \bftt{article}.
\item Un signe de percentatge \keystrokebftt{\%} inicia un \emph{comentari} --- \LaTeX{}
 ignorarà la resta de la línia.
\end{itemize}
\end{frame}

%%%%%%%%%%%%%%%%%%%%%%%%%%%%%%%%%%%%%%%%%%%%%%%%%%%%%%%%%%%%%%%%%%%%%%%%%%%%%%%
%%%%%%%%%%%%%%%%%%%%%%%%%%%%%%%%%%%%%%%%%%%%%%%%%%%%%%%%%%%%%%%%%%%%%%%%%%%%%%%
%%%%%%%%%%%%%%%%%%%%%%%%%%%%%%%%%%%%%%%%%%%%%%%%%%%%%%%%%%%%%%%%%%%%%%%%%%%%%%%
\begin{frame}[fragile]{\wllogo}
\begin{itemize}
\item Overleaf és un lloc web per escriure documents a \LaTeX.
\item `Compila' el teu \LaTeX{}  automàticament\footnote{Ctrl + Enter, o premeu `Recompile'} per a mostrar-te els resultats. 
\vskip 2em
\begin{center}
\fbox{\href{\wlnewdoc{basics.tex}}{%
Feu clic aquí per a obrir el document d'exemple a \wllogo{}}}
\\[1ex]\scriptsize{}
%Per obtenir els millors resultats, utilitzeu Google Chrome o un FireFox recent.
\end{center}
\vskip 2ex
\item A mesura que anem passant per les següents diapositives, proveu els exemples escrivint-los en el document d'exemple sobre Overleaf.
\item \textbf{Vinga, feu-ho a mesura que anem avançant!}
\end{itemize}
\end{frame}

%%%%%%%%%%%%%%%%%%%%%%%%%%%%%%%%%%%%%%%%%%%%%%%%%%%%%%%%%%%%%%%%%%%%%%%%%%%%%%%
%%%%%%%%%%%%%%%%%%%%%%%%%%%%%%%%%%%%%%%%%%%%%%%%%%%%%%%%%%%%%%%%%%%%%%%%%%%%%%%
%%%%%%%%%%%%%%%%%%%%%%%%%%%%%%%%%%%%%%%%%%%%%%%%%%%%%%%%%%%%%%%%%%%%%%%%%%%%%%%
\subsection{Escriure Text}
\begin{frame}[fragile]{Escriure Text}
\small
\begin{itemize}
\item Escriviu el text entre \cmdbegin{document} i \cmdend{document}.
\item En la seva major part, només has d'escriure el text amb normalitat.

\begin{itemize}
    \item Les paraules estan separades per \textbf{com a mínim} un espai.
    \item Els paràgrafs estan separats per \textbf{almenys} una línia en blanc.
\end{itemize}

\item Els espais del fitxer d'origen es co"lapsen a la sortida. 
\begin{exampletwouptiny}
La          fonologia
    estudia els fonemes 
\end{exampletwouptiny}
\item Pot ser útil començar cada frase en una nova línia (sobretot amb \href{https://git-scm.com/}{Git}).
\end{itemize}
\end{frame}

%%%%%%%%%%%%%%%%%%%%%%%%%%%%%%%%%%%%%%%%%%%%%%%%%%%%%%%%%%%%%%%%%%%%%%%%%%%%%%%
%%%%%%%%%%%%%%%%%%%%%%%%%%%%%%%%%%%%%%%%%%%%%%%%%%%%%%%%%%%%%%%%%%%%%%%%%%%%%%%
%%%%%%%%%%%%%%%%%%%%%%%%%%%%%%%%%%%%%%%%%%%%%%%%%%%%%%%%%%%%%%%%%%%%%%%%%%%%%%%
\begin{frame}[fragile]{Escriure Text: Compte}
\small
\begin{itemize}
\item Les cometes són una mica complicades:\\
useu un accent obert (\textit{backtick}) \keystroke{\`} a l'esquerra i un apòstrof (cometes simples) \keystroke{'} a la dreta.
\begin{exampletwouptiny}
Cometes simples: `text'.

Cometes dobles: ``text''.
\end{exampletwouptiny}

\item Alguns caràcters comuns tenen significats especials a \LaTeX:\\[1ex]
\begin{tabular}{cl}
\keystrokebftt{\%} & signe percentatge         \\
\keystrokebftt{\#} & coixinet (sostingut)      \\
\keystrokebftt{\&} & et (ampersand)            \\
\keystrokebftt{\$} & signe de dòlar            \\
\end{tabular}
\item Si només els escriviu, segurament obtindreu un error. 
    Si voleu que apareguin a la sortida, heu d'\textit{escapar-lo} precedint-lo amb una barra inversa \keystroke{\textbackslash}.
\begin{exampletwoup}
\$, \%, \&, \#
\end{exampletwoup}
\end{itemize}
\end{frame}

%%%%%%%%%%%%%%%%%%%%%%%%%%%%%%%%%%%%%%%%%%%%%%%%%%%%%%%%%%%%%%%%%%%%%%%%%%%%%%%
%%%%%%%%%%%%%%%%%%%%%%%%%%%%%%%%%%%%%%%%%%%%%%%%%%%%%%%%%%%%%%%%%%%%%%%%%%%%%%%
%%%%%%%%%%%%%%%%%%%%%%%%%%%%%%%%%%%%%%%%%%%%%%%%%%%%%%%%%%%%%%%%%%%%%%%%%%%%%%%
\begin{frame}[fragile]{Com Tractar els Errors}
\begin{itemize}
\item \LaTeX{} es pot confondre quan està intentant compilar el vostre document. 
    Si ho fa, s'atura amb un error, que heu de solucionar abans de produir cap sortida.
\item Per exemple, si escriviu malament \cmdbs{textbf} com a \cmdbs{textfb}, \LaTeX{} s'aturarà amb un error 
``undefined control sequence'', perquè ``textfb'' no és una de les ordres que coneix.
\end{itemize}
\begin{block}{Consells sobre Errors}
\begin{enumerate}
\item No t'espantis! És comú trobar errors.
\item Arregle'ls tan aviat com apareguin, si el que acabes d'escriure ha provocat un error, comença per allà. 
\item Si hi ha diversos errors, comença pel primer:
la causa fins i tot pot estar per sobre.
\end{enumerate}
\end{block}
\end{frame}

%%%%%%%%%%%%%%%%%%%%%%%%%%%%%%%%%%%%%%%%%%%%%%%%%%%%%%%%%%%%%%%%%%%%%%%%%%%%%%%
%%%%%%%%%%%%%%%%%%%%%%%%%%%%%%%%%%%%%%%%%%%%%%%%%%%%%%%%%%%%%%%%%%%%%%%%%%%%%%%
%%%%%%%%%%%%%%%%%%%%%%%%%%%%%%%%%%%%%%%%%%%%%%%%%%%%%%%%%%%%%%%%%%%%%%%%%%%%%%%
\begin{frame}[fragile]{Exercici 1}

\begin{block}{Escriu això amb \LaTeX:
\footnote{\url{http://en.wikipedia.org/wiki/Economy_of_the_United_States}}}
In March 2006, Congress raised that ceiling an additional \$0.79 trillion to
\$8.97 trillion, which is approximately 68\% of GDP. As of October 4, 2008, the
``Emergency Economic Stabilization Act of 2008'' raised the current debt ceiling
to \$11.3 trillion.
\end{block}
\vskip 2ex
\begin{center}
\fbox{\href{\wlnewdoc{basics-exercise-1.tex}}{%
Clica per obrir l'exercici a \wllogo{}}}
\end{center}

\begin{itemize}
\item Pista: vigila amb els caràcters amb significat especial! 
\item Un cop ho hagis provat,
\fbox{\href{\wlnewdoc{basics-exercise-1-solution.tex}}{%
clica aquí per veure la solució}}.
\end{itemize}
\end{frame}

%%%%%%%%%%%%%%%%%%%%%%%%%%%%%%%%%%%%%%%%%%%%%%%%%%%%%%%%%%%%%%%%%%%%%%%%%%%%%%%
%%%%%%%%%%%%%%%%%%%%%%%%%%%%%%%%%%%%%%%%%%%%%%%%%%%%%%%%%%%%%%%%%%%%%%%%%%%%%%%
%%%%%%%%%%%%%%%%%%%%%%%%%%%%%%%%%%%%%%%%%%%%%%%%%%%%%%%%%%%%%%%%%%%%%%%%%%%%%%%
\subsection{Escriure Matemàtiques}
\begin{frame}[fragile]{Escriure Matemàtiques: Símbols de dòlar}
\begin{itemize}
\item Per què són especials els símbols de dòlar \keystrokebftt{\$}? Els utilitzem per marcar matemàtiques dins el text.\\[1ex]
\begin{exampletwouptiny}
% meh:
Siguin a i b nombres enters 
positius, i sigui c > a - b + 1

% millor:
Siguin $a$ i $b$ nombres enters 
positius, i sigui $c > a - b + 1$.
\end{exampletwouptiny}
\item Utilitza sempre els símbols de dòlar amb pareller: un per començar les matemàtiques, i un per acabar-les.
\item \LaTeX{} ignora els espais i els ajusta automàticament.
\begin{exampletwouptiny}
$y=ax^2+bx+c$ \dots

$y = a x^2 + b x + c $ \dots
\end{exampletwouptiny}
\end{itemize}
\end{frame}

%%%%%%%%%%%%%%%%%%%%%%%%%%%%%%%%%%%%%%%%%%%%%%%%%%%%%%%%%%%%%%%%%%%%%%%%%%%%%%%
%%%%%%%%%%%%%%%%%%%%%%%%%%%%%%%%%%%%%%%%%%%%%%%%%%%%%%%%%%%%%%%%%%%%%%%%%%%%%%%
%%%%%%%%%%%%%%%%%%%%%%%%%%%%%%%%%%%%%%%%%%%%%%%%%%%%%%%%%%%%%%%%%%%%%%%%%%%%%%%
\begin{frame}[fragile]{Escriure Matemàtiques: Notació}
\begin{itemize}
\item Accent circumflex \keystrokebftt{\^} per superíndexs i barra baixa \keystrokebftt{\_} pels subíndexs.
\begin{exampletwouptiny}
$y = c_2 x^2 + c_1 x + c_0$
\end{exampletwouptiny}
\vskip 2ex

\item Useu claus \keystrokebftt{\{} \keystrokebftt{\}} per agrupar els índexs.
\begin{exampletwouptiny}
$F_n = F_n-1 + F_n-2$     % oops!

$F_n = F_{n-1} + F_{n-2}$ % ok!
\end{exampletwouptiny}
\vskip 2ex

\item Hi ha ordres per a les lletres gregues en notació comuna. 
\begin{exampletwouptiny}
$\mu = A e^{Q/RT}$

$\Omega = \sum_{k=1}^{n} \omega_k$
\end{exampletwouptiny}
\end{itemize}
\end{frame}

%%%%%%%%%%%%%%%%%%%%%%%%%%%%%%%%%%%%%%%%%%%%%%%%%%%%%%%%%%%%%%%%%%%%%%%%%%%%%%%
%%%%%%%%%%%%%%%%%%%%%%%%%%%%%%%%%%%%%%%%%%%%%%%%%%%%%%%%%%%%%%%%%%%%%%%%%%%%%%%
%%%%%%%%%%%%%%%%%%%%%%%%%%%%%%%%%%%%%%%%%%%%%%%%%%%%%%%%%%%%%%%%%%%%%%%%%%%%%%%
% \begin{frame}[fragile]{\insertsubsection{}: Displayed Equations}
% \begin{itemize}
% \item If it's big i scary, \emph{display} it on its own line using
% \cmdbegin{equation} i \cmdend{equation}.\\[2ex]
% \begin{exampletwouptiny}
% The roots of a quadratic equation
% are given by
% \begin{equation}
% x = \frac{-b \pm \sqrt{b^2 - 4ac}}
%          {2a}
% \end{equation}
% where $a$, $b$ i $c$ are \dots
% \end{exampletwouptiny}
% \vskip 1em
% {\scriptsize Caution: \LaTeX{} mostly ignores your spaces in mathematics, but it
% can't handle blank lines in equations --- don't put blank lines in your
% mathematics.}
% \end{itemize}
% \end{frame}


%%%%%%%%%%%%%%%%%%%%%%%%%%%%%%%%%%%%%%%%%%%%%%%%%%%%%%%%%%%%%%%%%%%%%%%%%%%%%%%
%%%%%%%%%%%%%%%%%%%%%%%%%%%%%%%%%%%%%%%%%%%%%%%%%%%%%%%%%%%%%%%%%%%%%%%%%%%%%%%
%%%%%%%%%%%%%%%%%%%%%%%%%%%%%%%%%%%%%%%%%%%%%%%%%%%%%%%%%%%%%%%%%%%%%%%%%%%%%%%
\begin{frame}[fragile]{Entorns}
\begin{itemize}
\item Les ordres \cmdbs{begin} i \cmdbs{end} es fan servir per delimitar diferents entorns.
\vskip 2ex

\item Els entorns \bftt{itemize} i \bftt{enumerate} generen llistes de punts i numèriques. 
\begin{exampletwouptiny}
\begin{itemize} % per punts 
    \item Galetes
    \item Iogurt
\end{itemize}

\begin{enumerate} % per nombres
    \item Galetes
    \item Iogurt
\end{enumerate}
\end{exampletwouptiny}
\item No és necessari, però és bona pràctica augmentar el sagnat del text dins d'un entorn. 
\end{itemize}
\end{frame}

%%%%%%%%%%%%%%%%%%%%%%%%%%%%%%%%%%%%%%%%%%%%%%%%%%%%%%%%%%%%%%%%%%%%%%%%%%%%%%%
%%%%%%%%%%%%%%%%%%%%%%%%%%%%%%%%%%%%%%%%%%%%%%%%%%%%%%%%%%%%%%%%%%%%%%%%%%%%%%%
%%%%%%%%%%%%%%%%%%%%%%%%%%%%%%%%%%%%%%%%%%%%%%%%%%%%%%%%%%%%%%%%%%%%%%%%%%%%%%%
\begin{frame}[fragile]{Entorns 2 (extra)}
\begin{itemize}
\item \bftt{equation} és l'entorn més comú per posar equacions numerades.
\item Les mateixes ordres dins un entorn produeixen resultats diferents. 
\begin{exampletwouptiny}
Podem escriure el mateix dins un text: 
$ \Omega = \sum_{k=1}^{n} \omega_k $

O dir, com es veu a:
\begin{equation}
  \Omega = \sum_{k=1}^{n} \omega_k
\end{equation}
\end{exampletwouptiny}
\vskip 2ex
\item Fixeu-vos que $\Sigma$ és més gran dins d'\bftt{equation}, i els subíndexs i superíndex canvien de lloc.
\vskip 1em
{\scriptsize De fet, \bftt{\$...\$} és una drecera per l'entorn
\cmdbegin{math}\bftt{...}\cmdend{math}.}
\end{itemize}
\end{frame}
%%%%%%%%%%%%%%%%%%%%%%%%%%%%%%%%%%%%%%%%%%%%%%%%%%%%%%%%%%%%%%%%%%%%%%%%%%%%%%%
%%%%%%%%%%%%%%%%%%%%%%%%%%%%%%%%%%%%%%%%%%%%%%%%%%%%%%%%%%%%%%%%%%%%%%%%%%%%%%%
%%%%%%%%%%%%%%%%%%%%%%%%%%%%%%%%%%%%%%%%%%%%%%%%%%%%%%%%%%%%%%%%%%%%%%%%%%%%%%%
\begin{frame}[fragile]{Incís: Paquets}

\begin{itemize}
\item Totes les ordres i entorns que hem vist fins ara venen integrades per defecte amb \LaTeX.

\item Els \emph{paquets} són llibreries que contenen ordres i entorns addicionals. N'hi ha milers de gratuïts.

\item Si volem fer servir un paquet l'hem de carregar amb l'ordre 
    \cmdbs{usepackage} al \emph{preàmbul} \footnote{Un dels beneficis d'Overleaf és que ja té tots els paquets de \href{https://tug.org/texlive/}{TexLive} insta"lats.}.

\item Exemple: \bftt{linguex} per exemples lingüístics. 
\begin{minted}[fontsize=\small,frame=single]{latex}
\documentclass{article}
\usepackage{linguex} % preàmbul
\begin{document}
% ja podem fer servir ordres de linguex ... 
\end{document}
\end{minted}
\end{itemize}
\end{frame}

%%%%%%%%%%%%%%%%%%%%%%%%%%%%%%%%%%%%%%%%%%%%%%%%%%%%%%%%%%%%%%%%%%%%%%%%%%%%%%%
%%%%%%%%%%%%%%%%%%%%%%%%%%%%%%%%%%%%%%%%%%%%%%%%%%%%%%%%%%%%%%%%%%%%%%%%%%%%%%%
%%%%%%%%%%%%%%%%%%%%%%%%%%%%%%%%%%%%%%%%%%%%%%%%%%%%%%%%%%%%%%%%%%%%%%%%%%%%%%%
\subsection{linguex}
\begin{frame}[fragile]{linguex: exemples}
\begin{itemize}
\item Les 3 ordres bàsiques de linguex són \cmdbs{ex.}, \cmdbs{a.} i \cmdbs{b.}. 
\item \cmdbs{ex.} inicia l'exemple, i cal una línia en blanc per acabar-lo.
\begin{exampletwouptiny2}
\ex. Exemple

\ex. Primer nivell de l'exemple
\a. Segon nivell de l'exemple
\b. Seguim al segon nivell 

Com hem vist a \LLast i \Last \dots

En canvi a \Next veiem \dots

\ex. 
\a. podem deixar buit 
\b. el primer nivell
\c. si volem ser ordenats
\d. podem fer servir
\e. tot l'abecedari
\b. tot i que no cal

\end{exampletwouptiny2}
\item Les ordres \cmdbs{c.}, \cmdbs{d.}, etc.~són còpies de \cmdbs{b.}.
\end{itemize}
\end{frame}

%%%%%%%%%%%%%%%%%%%%%%%%%%%%%%%%%%%%%%%%%%%%%%%%%%%%%%%%%%%%%%%%%%%%%%%%%%%%%%%
%%%%%%%%%%%%%%%%%%%%%%%%%%%%%%%%%%%%%%%%%%%%%%%%%%%%%%%%%%%%%%%%%%%%%%%%%%%%%%%
%%%%%%%%%%%%%%%%%%%%%%%%%%%%%%%%%%%%%%%%%%%%%%%%%%%%%%%%%%%%%%%%%%%%%%%%%%%%%%%
\begin{frame}[fragile]{linguex: exemples 2}
\begin{itemize}
\item Per acabar un sol nivell s'utilitza \cmdbs{z.} 
\begin{exampletwouptiny2}
\ex. Animals 
\a. Gats
\a. miau
\z.
\b. Gossos
\a. bup
\z.
\b. Fures

\end{exampletwouptiny2}
\end{itemize}

\end{frame}

%%%%%%%%%%%%%%%%%%%%%%%%%%%%%%%%%%%%%%%%%%%%%%%%%%%%%%%%%%%%%%%%%%%%%%%%%%%%%%%
%%%%%%%%%%%%%%%%%%%%%%%%%%%%%%%%%%%%%%%%%%%%%%%%%%%%%%%%%%%%%%%%%%%%%%%%%%%%%%%
%%%%%%%%%%%%%%%%%%%%%%%%%%%%%%%%%%%%%%%%%%%%%%%%%%%%%%%%%%%%%%%%%%%%%%%%%%%%%%%
\begin{frame}[fragile]{linguex: gramaticalitat}
\begin{itemize}
    \item Linguex admet els símbols *, ?, \# i \% per judicis de gramaticalitat (els posa al davant)\footnote{recordeu que \# i \% s'escriuen amb una \bs~al davant}.
\begin{exampletwouptiny2}
\ex. 
\a. Exemple ben format 
\b. * Agramatical frase? 
\b. ** i molt agramatical molt
\b. \# El formatge l'hi he posat 
\c. ? Hi han maduixes 

\end{exampletwouptiny2}
\end{itemize}

\end{frame}

%%%%%%%%%%%%%%%%%%%%%%%%%%%%%%%%%%%%%%%%%%%%%%%%%%%%%%%%%%%%%%%%%%%%%%%%%%%%%%%
%%%%%%%%%%%%%%%%%%%%%%%%%%%%%%%%%%%%%%%%%%%%%%%%%%%%%%%%%%%%%%%%%%%%%%%%%%%%%%%
%%%%%%%%%%%%%%%%%%%%%%%%%%%%%%%%%%%%%%%%%%%%%%%%%%%%%%%%%%%%%%%%%%%%%%%%%%%%%%%
\begin{frame}[fragile]{linguex: glosses}
\begin{itemize}
\item Linguex és molt últil per morfologia. 
\item Per glossar un exemple afegim `g' després de l'última lletra de l'ordre i linguex s'encarrega d'alinear les paraules (cal \keystrokebftt{\bs\bs}  per marcar el final de línia).
\begin{exampletwouptiny2}
\ex.\a. No gloss
\bg. This is a first gloss\\
Dies ist eine erste Glosse\\

\exg.
Dies ist nicht die erste Glosse\\
This is not the first gloss\\

\end{exampletwouptiny2}
\item Podem afegir una tercera línia de traducció amb \cmdbs{glt} (gloss translation).
\begin{exampletwouptiny2}
\exg. 
Martin-ek Diego-\zero{} ikusi du \\
Martin-ERG Diego-ABS vist ha \\
\glt `En Martin ha vist en Diego'

\end{exampletwouptiny2}
\end{itemize}

\end{frame}

%%%%%%%%%%%%%%%%%%%%%%%%%%%%%%%%%%%%%%%%%%%%%%%%%%%%%%%%%%%%%%%%%%%%%%%%%%%%%%%
%%%%%%%%%%%%%%%%%%%%%%%%%%%%%%%%%%%%%%%%%%%%%%%%%%%%%%%%%%%%%%%%%%%%%%%%%%%%%%%
%%%%%%%%%%%%%%%%%%%%%%%%%%%%%%%%%%%%%%%%%%%%%%%%%%%%%%%%%%%%%%%%%%%%%%%%%%%%%%%
\begin{frame}[fragile]{linguex: glosses 2}
\begin{itemize}
\item Podem modificar les glosses perquè tinguin l'aspecte que vulguem.
\begin{exampletwouptiny}
\renewcommand{\eachwordone}{\itshape}
\renewcommand{\eachwordtwo}{\tiny}

\exg. 
Martin-ek Diego-\zero{} ikusi du \\
Martin-ERG Diego-ABS vist ha \\
\glt `En Martin ha vist en Diego'


\end{exampletwouptiny}
\end{itemize}

\end{frame}

%%%%%%%%%%%%%%%%%%%%%%%%%%%%%%%%%%%%%%%%%%%%%%%%%%%%%%%%%%%%%%%%%%%%%%%%%%%%%%%
%%%%%%%%%%%%%%%%%%%%%%%%%%%%%%%%%%%%%%%%%%%%%%%%%%%%%%%%%%%%%%%%%%%%%%%%%%%%%%%
%%%%%%%%%%%%%%%%%%%%%%%%%%%%%%%%%%%%%%%%%%%%%%%%%%%%%%%%%%%%%%%%%%%%%%%%%%%%%%%
\begin{frame}[fragile]{linguex: claudators}
\begin{itemize}
\item Amb l'ordre \cmdbs{exi.} es pot fer servir notació amb claudators
\item Si l'exemple conté una glossa, cal saltar-se elements amb `\{\}'
\begin{MyMinted}
\exi.
\a.  [SC that [ST John$_i$  [Sv $t_i$ likes Mary ]]]
\bg. [SC dass [ST Peter$_i$ [Sv $t_i$ Mari liebt ]]] \\
{} that {} Peter {} {} Mary loves \\
\glt `that Peter loves Mary'
\end{MyMinted}
\exi.
\a.  [SC that [ST John$_i$  [Sv $t_i$ likes Mary ]]]
\bg. [SC dass [ST Peter$_i$ [Sv $t_i$ Mari liebt ]]] \\
{} that {} Peter {} {} Mary loves \\
\glt `that Peter loves Mary'

\end{itemize}

\end{frame}

%%%%%%%%%%%%%%%%%%%%%%%%%%%%%%%%%%%%%%%%%%%%%%%%%%%%%%%%%%%%%%%%%%%%%%%%%%%%%%%
%%%%%%%%%%%%%%%%%%%%%%%%%%%%%%%%%%%%%%%%%%%%%%%%%%%%%%%%%%%%%%%%%%%%%%%%%%%%%%%
%%%%%%%%%%%%%%%%%%%%%%%%%%%%%%%%%%%%%%%%%%%%%%%%%%%%%%%%%%%%%%%%%%%%%%%%%%%%%%%
\subsection{Extra}
\begin{frame}[fragile]{Extra: arbres}
\begin{itemize}
\item Hi ha dos paquets importants per fer arbres: forest i tikz-qtree.  
\item Mireu els manuals per saber quin uns convé.
\begin{exampletwouptiny}
\ex. \begin{forest}
[SC[C][ST[T][SV[V][SN]]]]
\end{forest}

\end{exampletwouptiny}
\item \href{https://ling.auf.net/lingbuzz/003391}{Guia forest}
\end{itemize}

\end{frame}
%%%%%%%%%%%%%%%%%%%%%%%%%%%%%%%%%%%%%%%%%%%%%%%%%%%%%%%%%%%%%%%%%%%%%%%%%%%%%%%
%%%%%%%%%%%%%%%%%%%%%%%%%%%%%%%%%%%%%%%%%%%%%%%%%%%%%%%%%%%%%%%%%%%%%%%%%%%%%%%
%%%%%%%%%%%%%%%%%%%%%%%%%%%%%%%%%%%%%%%%%%%%%%%%%%%%%%%%%%%%%%%%%%%%%%%%%%%%%%%
\begin{frame}[fragile]{Extra: Examples with \bftt{amsmath}}
\begin{itemize}
\item Use \bftt{equation*} (``equation-star'') for unnumbered equations.
\begin{exampletwouptiny}
\begin{equation*}
  \Omega = \sum_{k=1}^{n} \omega_k
\end{equation*}
\end{exampletwouptiny}
\item \LaTeX{} treats adjacent letters as variables multiplied together, which
is not always what you want. \bftt{amsmath} defines ordres for many common
mathematical operators.
\begin{exampletwouptiny}
\begin{equation*} % bad!
 min_{x,y} (1-x)^2 + 100(y-x^2)^2
\end{equation*}
\begin{equation*} % good!
\min_{x,y}{(1-x)^2 + 100(y-x^2)^2}
\end{equation*}
\end{exampletwouptiny}
\item You can use \cmdbs{operatorname} for others.
\begin{exampletwouptiny}
\begin{equation*}
\beta_i =
\frac{\operatorname{Cov}(R_i, R_m)}
     {\operatorname{Var}(R_m)}
\end{equation*}
\end{exampletwouptiny}
\end{itemize}
\end{frame}

%%%%%%%%%%%%%%%%%%%%%%%%%%%%%%%%%%%%%%%%%%%%%%%%%%%%%%%%%%%%%%%%%%%%%%%%%%%%%%%
%%%%%%%%%%%%%%%%%%%%%%%%%%%%%%%%%%%%%%%%%%%%%%%%%%%%%%%%%%%%%%%%%%%%%%%%%%%%%%%
%%%%%%%%%%%%%%%%%%%%%%%%%%%%%%%%%%%%%%%%%%%%%%%%%%%%%%%%%%%%%%%%%%%%%%%%%%%%%%%
\begin{frame}[fragile]{Extra: Examples with \bftt{amsmath}}
\begin{itemize}{\small
\item Align a sequence of equations at the equals sign
\begin{align*}
(x+1)^3 &= (x+1)(x+1)(x+1) \\
        &= (x+1)(x^2 + 2x + 1) \\
        &= x^3 + 3x^2 + 3x + 1
\end{align*}
with the \bftt{align*} environment.

% for whatever reason, this doesn't play well with the twoup environment
\begin{minted}[fontsize=\small,frame=single]{latex}
\begin{align*}
(x+1)^3 &= (x+1)(x+1)(x+1) \\
        &= (x+1)(x^2 + 2x + 1) \\
        &= x^3 + 3x^2 + 3x + 1
\end{align*}
\end{minted}
\item An ampersand \keystrokebftt{\&} separates the left column (before the
$=$) from the right column (after the $=$).
\item A double backslash \keystrokebftt{\bs}\keystrokebftt{\bs} starts a new
line.
}\end{itemize}
\end{frame}


%%%%%%%%%%%%%%%%%%%%%%%%%%%%%%%%%%%%%%%%%%%%%%%%%%%%%%%%%%%%%%%%%%%%%%%%%%%%%%%
%%%%%%%%%%%%%%%%%%%%%%%%%%%%%%%%%%%%%%%%%%%%%%%%%%%%%%%%%%%%%%%%%%%%%%%%%%%%%%%
%%%%%%%%%%%%%%%%%%%%%%%%%%%%%%%%%%%%%%%%%%%%%%%%%%%%%%%%%%%%%%%%%%%%%%%%%%%%%%%
\begin{frame}[fragile]{Typesetting Exercise 2}

\begin{block}{Typeset this in \LaTeX:}
Let $X_1, X_2, \dots, X_n$ be a sequence of independent i identically
distributed random variables with $\operatorname{E}[X_i] = \mu$ i
$\operatorname{Var}[X_i] = \sigma^2 < \infty$, i let
\begin{equation*}
S_n = \frac{1}{n}\sum_{i=1}^{n} X_i
\end{equation*}
denote their mean. Then as $n$ approaches infinity, the random variables
$\sqrt{n}(S_n - \mu)$ converge in distribution to a normal $N(0, \sigma^2)$.
\end{block}
\vskip 2ex
\begin{center}
\fbox{\href{\wlnewdoc{basics-exercise-2.tex}}{%
Click to open this exercise in \wllogo{}}}
\end{center}
\begin{itemize}
\item Hint: the ordre for $\infty$ is \cmdbs{infty}.
\item Once you've tried,
\fbox{\href{\wlnewdoc{basics-exercise-2-solution.tex}}{%
click here to see my solution}}.
\end{itemize}
\end{frame}

%%%%%%%%%%%%%%%%%%%%%%%%%%%%%%%%%%%%%%%%%%%%%%%%%%%%%%%%%%%%%%%%%%%%%%%%%%%%%%%
%%%%%%%%%%%%%%%%%%%%%%%%%%%%%%%%%%%%%%%%%%%%%%%%%%%%%%%%%%%%%%%%%%%%%%%%%%%%%%%
%%%%%%%%%%%%%%%%%%%%%%%%%%%%%%%%%%%%%%%%%%%%%%%%%%%%%%%%%%%%%%%%%%%%%%%%%%%%%%%
\begin{frame}{Final de la primera part}
\begin{itemize}
\item Ja hem aprés \dots 
\begin{itemize}
\item Escriure text amb \LaTeX.
\item Utilitzar diferents ordres.
\item Tractar errors quan apareixen. 
\item Escriure matemàtiques. 
\item Utilitzar diversos entorns diferents.
\item Carregar paquets.
\end{itemize}
\item A la segona part, veurem com utilitzar \LaTeX{} per escriure documents amb seccions, referències creuades, figures, taules i bibliografia. Fins llavors!
\end{itemize}
\end{frame}

\end{document}
