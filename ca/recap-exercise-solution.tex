\documentclass[12pt]{article}
\usepackage[catalan]{babel}

\usepackage{url}

\title{Deu secrets per fer una bona xerrada científica}

\author{Jo}
\date{\today}

\begin{document}
\maketitle

\section{Introducció}

El text d'aquest exercici és una versió significativament abreujada i lleugerament modificada de l'excel·lent article del mateix nom de Mark Schoeberl i Brian Toon:
\url{http://www.cgd.ucar.edu/cms/agu/scientific_talk.html}

\section{Els secretes}

He recopilat aquesta llista personal de ``Secrets'' d'escoltar oradors efectius i ineficaços. No pretenc que aquesta llista sigui exhaustiva --- estic segur que hi ha coses que he deixat fora. Però la meva llista probablement cobreix aproximadament el 90\% del que necessites saber i fer.

\begin{enumerate}

\item Prepara el teu material amb cura i lògicament. Explica una història.

\item Practica la teva xerrada. No hi ha excusa per a la falta de preparació.

\item No posis massa material. Els bons oradors tindran un o dos punts centrals i s'adheriran a aquest material.

\item Evita equacions. Es diu que per a cada equació de la seva xerrada, el nombre de persones que ho entendran es reduirà a la meitat. És a dir, si deixem que q sigui el nombre d'equacions de la vostra xerrada i n sigui el nombre de persones que entenguin la vostra xerrada, sobté que

\begin{equation}
n = \gamma \left( \frac{1}{2} \right)^q
\end{equation}

on $\gamma$ és una constant de proporcionalitat.

\item Tenir només alguns punts de conclusió. La gent no pot recordar més d'un parell de coses d'una xerrada, especialment si estan sentint moltes xerrades en reunions grans.

\item Parla amb el públic, no a la pantalla. Un dels problemes més freqüents que veig és que l'orador parlarà a la pantalla del visor.

\item Evita fer sons distraients. Intenteu evitar ``Ummm'' o ``Ahhh'' entre frases.

\item Puleix els teus gràfics. Aquí hi ha una llista de consells per a millors gràfics:

\begin{itemize}

    \item Utilitza lletres grans.

    \item Mantingues els gràfics senzills. No mostris els gràfics que no necessitaràs.

    \item Usa color.

\end{itemize}

\item Sigues àgil responent preguntes.

\item Utilitza l'humor si és possible. Sempre em sorprèn com fins i tot una broma realment absurda farà riure en una xerrada científica.

\end{enumerate}

\end{document}
